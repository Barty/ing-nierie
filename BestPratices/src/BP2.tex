  
\twocolumn[
\section{Best Pratice 2 : Rédaction d'un cahier des charges}
Un cahier des charges vise à définir simplement les \textbf{"spécifications de base"} d’un produit ou d’un service à réaliser.
\textbf{C'est définir le besoin du client.\\~\\}
                  
]{
\setlength{\columnsep}{0.3cm}

\subsection{Plan type}
      \noitemsep
      \small{
    \begin{enumerate}

      \item  Introduction
        \begin{enumerate}
          \item Présentation du projet
          \item Présentation du document 
          \item Documents applicables / Documents de référence
          \item Terminologie et abréviations
        \end{enumerate}  
      \item  Présentation du problème
        \begin{enumerate}
          \item But
          \item Formulation des besoins, exploitation et ergonomie, expérience 
          \item Portée, développement, mise en oeuvre, organisation de la maintenance
          \item Limites
        \end{enumerate} 
      \item  Exigences fonctionnelles
        \begin{enumerate}
          \item Fonction de base, performances et aptitudes
          \item Contraintes d'utilisation 
          \item Critères d'appréciation de la réalisation effective de la fonction
          \item Flexibilité dans la façon de mettre en oeuvre la fonction concernée et variation de coûts associée en fonction de cette flexibilité
        \end{enumerate} 
      \item Contraintes imposées, faisabilité technologique et éventuellement moyens
        \begin{enumerate}
          \item Sûreté, planning, organisation, communication
          \item Complexité 
          \item Compétences, moyens et règles
          \item Normes de documentation
        \end{enumerate} 
      \item  Configuration cible
        \begin{enumerate}
          \item Matériel et logiciels
          \item Stabilité de la configuration
          \item (Description des API)
        \end{enumerate} 
      \item  Guide de réponse au cahier des charges
        \begin{enumerate}
          \item Grille d'évaluation
        \end{enumerate} 
      \item  Annexes (liste à titre d'exemples)
        \begin{enumerate}
          \item Observations de l'existant
          \item Propositions d'orientation
          \item Image(s) d'écran(s) principaux du logiciel
          \item Résultat de l'analyse de la valeur
          \item Description des API
          \item Choix d'une solution et justifications
          \item Appréciation de la solution retenue 
        \end{enumerate}         
        
      \end{enumerate}
        }
       
        \doitemsep
\subsubsection{Objectifs du Cahier des Charges}
  \begin{itemize}
    \item Exprimer les objectifs, les besoins et les exigences de l'utilisateur
    \item Définir les caractéristiques et le champ d'application du produit à acquérir
    \item Répondre à la question "Que doit faire le produit/service?"
  \end{itemize}
\subsubsection{Bonnes Pratiques}
   \begin{itemize}
    \item Le CdC doit être le plus exhaustif possible
    \item Une question doit être précise et compréhensible
    \item Plus les réponses à une question sont nombreuses et quantifiées, plus l'évaluation est simple
    \item Le processus d'évaluation doit si possible être informatisé
    \item Le CdC traduit les obligations de résultat et non les exigences de moyens
    \item Comme pour tout, Un beau dessin vaut mieux qu’un long discours (faire des schémas)
    \item utiliser un langage général, compréhensible de tous
    \item Utiliser des verbes clef
    \item Utiliser le futur, afin de positionner le CdC dans le processus global
  \end{itemize}
  \subsubsection{Documents de Référence}
  \begin{itemize}
\item Document "exemple de procedure pour la redaction d'un CdC" présent dans servif-baie.
\item Cours de Génie Logiciel 3IF, Chapitre II
\end{itemize}       
}

   