\usepackage[english, french]{babel}
\usepackage[utf8]{inputenc}

\usepackage{graphicx}
\usepackage{fancyhdr}
\usepackage{lastpage}
\usepackage{amsmath}
\usepackage{xspace}
\usepackage{textcomp}

\usepackage[colorlinks=false,breaklinks=true]{hyperref}

\usepackage[top=20mm, bottom=20mm, left=25mm, right=25mm]{geometry}

\pagestyle{fancy}

\usepackage{helvet}
\usepackage{bbm}


\usepackage{noitemspace}

\usepackage{verbatim}
\usepackage{amsmath}
\usepackage[table]{xcolor}
\definecolor{bleugris}{rgb}{.2,.4,.5}

\definecolor{colKeys}{rgb}{0,0,1} 
\definecolor{colIdentifier}{rgb}{0,0,0} 
\definecolor{colComments}{rgb}{0,0.5,1} 
\definecolor{colString}{rgb}{0.6,0.1,0.1} 

\usepackage{listings}

% Permet l'ajout de code par insertion du fichier le contenant
% Les arguments sont :
% $1 : nom du fichier à inclure
% $2 : le type de langage (C++, C, Java ...)
\newcommand{\addCode}[2]{%

  % Configuration de la coloration syntaxique du code
  \definecolor{colKeys}{rgb}{0,0,1}
  \definecolor{colIdentifier}{rgb}{0,0,0}
  \definecolor{colComments}{rgb}{0,0.5,1}
  \definecolor{colString}{rgb}{0.6,0.1,0.1}

  % Configuration des options 
  \lstset{%
    language = #2,%
    identifierstyle=\color{colIdentifier},%
    basicstyle=\ttfamily\scriptsize, %
    keywordstyle=\color{colKeys},%
    stringstyle=\color{colString},%
    commentstyle=\color{colComments},%
    columns = flexible,%
    %tabsize = 8,%
    showspaces = false,%
    numbers = left, numberstyle=\tiny,%
    frame = single,frameround=tttt,%
    breaklines = true, breakautoindent = true,%
    captionpos = b,%
    xrightmargin=10mm, xleftmargin = 15mm, framexleftmargin = 7mm,%
  }%
    \begin{center}
    \lstinputlisting{#1}
    \end{center}
}

\newcommand{\nTitle}[1]{%
	\clearpage
	\vspace*{\fill}		%
	\begin{center}	%
		\part{#1}		%
	\end{center}
	\vspace*{\fill}		%
	\clearpage
}

\newenvironment{nAbstract} 		%
{ 								%
	\newpage 					% 
	\vspace*{\fill}				%
	\begin{center}			 	%
		\begin{abstract}		%
}{								%
		\end{abstract}			%
	\end{center}				%
	\vspace*{\fill}				%
	\newpage					%
}


\newcommand{\nClass}[1]{{\color{bleugris}{\textsl{\textbf{#1}}}}}
\newcommand{\nParameter}[1]{{\color{gray}{\textbf{#1}}}}
\newcommand{\nMethod}[1]{{\color{gray}{\textbf{#1}}}}
\newcommand{\nConstant}[1]{\texttt{\uppercase{#1}}}
\newcommand{\nKeyword}[1]{\textsl{\textbf{#1}}}

\graphicspath{{../SourcesMatlab/}}

% Conversion nombres arabes / romain
\makeatletter
\newcommand{\rmnum}[1]{\romannumeral #1}
\newcommand{\Rmnum}[1]{\expandafter\@slowromancap\romannumeral #1@}
\makeatother

\setlength{\headheight}{14pt}

\fancyhf{}
