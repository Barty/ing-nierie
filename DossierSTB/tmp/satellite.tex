\section{Prise en main de latex}
\subsection{Installation}
	\begin{description}
		\item[Linux]~\\
			\texttt{sudo aptitude install texlive-full}
		\item[Windows]~\\
			\href{http://mirror.ctan.org/systems/texlive/tlnet/install-tl.zip}{lien}
	\end{description}

\subsection{Compilation}
	Pour simplifier le processus, j'ai fais un Makefile qui automatise le processus.
	il peut être remplaçé par :\\
		\texttt{\$ cp src\/*.tex tmp\/}\\
		\texttt{\$ cd tmp\/}\\		
		\texttt{\$ pdflatex main.tex}\\
		\texttt{\$ mv main.pdf ..\/bin\/}\\

\section{Quelques exemples en latex}
\subsection{formattage de texte}

	\textbf{gras}\\
	\textit{italique}\\
	\emph{emphatique}\\
	\texttt{chasse fixe}\\

\subsection{Les listes}
		
	\begin{itemize}
	\item un item
	\item un deuxième item
		\subitem un sous item
	\item un troisème item
	\end{itemize}
