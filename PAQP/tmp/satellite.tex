\section{Préliminaires}
\subsection{Cadre du Plan d’Assurance Qualité Projet (ou PAQP}
  Le PAQP est mis en place dans le cadre de la réponse à l'appel d'offre "Système de monitoring à distance de sites isolés" par l'hexanôme H4212 (Maitrise d'oeuvre ou MOE) lancé par le COPEVUE.

\par Toute la documentation fournie par la MOE durant le projet est concernée par ce PAQP ainsi que tout sous-projet.

\subsection{Objectif du Plan d’Assurance Qualité Projet}
\subparagraph{La mise en place d'un système de qualité} tout au long de la conduite du projet. Ce PAQP doit être mis en oeuvre par la MOE afin de satisfaire la maitrise d'ouvrage.
\subparagraph{L'assurance de la cohérence et de l'homogénéité des documents} et livrables produits. le PAQP est mis en place pour assurer la qualité du produit fini.

\subsection{Responsabilités associées au Plan d’Assurance Qualité Projet}                        
L'ensemble des membres de la MOE est concerné par le PAQP. Pour la bonne conduite du projet, il est obligatoire que le PAQP soit connu de tous, et qu'il soit appliqué. Cependant, chaque personne a un rôle différent vis-à-vis du PAQP :

\paragraph{Groupe d'Etude Informatique (GEI)\\}	 
                \begin{itemize}
                \item Appliquer le PAQP
                \item Corriger les documents non-conformes pour être en conformité avec le PAQP
                \end{itemize}

\paragraph{Responsable Qualité (RQ)\\}
                \begin{itemize}
                \item Rédiger et améliorer le PAQP
                \item Garantir l'application du PAQP
                \end{itemize}	
                
\paragraph{Chef de Projet(CdP)\\} 
                \begin{itemize}
                \item Appliquer le PAQP
                \item Valider le PAQP
                \item Faire respecter le PAQP
                \end{itemize}
    
\subsection{Procédures d’évolution du PAQP}

Il appartient à chacun de faire évoluer le PAQP. Par définition, le PAQP est voué à évoluer pour s'améliorer afin d'atteindre le "Zero Defaut".
\par Le PAQP peut-être amené à évoluer pour plusieurs raisons : 
\begin{itemize}
\item détection d'un défaut ou d'une imprécision dans le PAQP 
\item découverte d'une "Best-Practice" qui peut être source d'inspiration pour le présent PAQP 
\item mise en place d'une nouvelle règle, d'un nouvel outil,...\\
\end{itemize}

Toute évolution doit être soumise au RQ, qui la prendra en considération, et qui devra être validée par le CdP. 
Lorsqu'une procédure d'évolution du PAQP aboutie, tous les membres du projet sont avertis et informés.

\subsection{Procédure à suivre en cas de non-application du PAQP}

Première régle importante : lorsqu'un document, résultat ou livrable produit par l'équipe du projet ne respecte pas le PAQP, il ne pourra pas être validé.

L'auteur de la non-conformité doit être averti par le RQ et/ou le CdP, et il lui sera fourni les éléments et informations nécessaires à la correction. Ce dernier devra alors prendre en compte ces informations, et procéder aux modifications nécessaires, pour que le document produit puisse être définitivement validé.

\subsection{Procédure de dérogation du PAQP}
Une certaine flexibilité dans l'application du PAQP est envisageable. Il ne s'agit d'être rigoriste.

De ce fait, si un membre de l'équipe du projet, pour un document, résultat ou livrable en cours de production, juge opportuniste pour des raisons données de ne pas appliquer des règles du PAQP, il peut en faire part au RQ, avec des justification.

En fonction des justifications, le RQ prend la décision d'accorder ou pas la dérogation. En cas de dérogation, il en averti le CdP. Si le membre du projet se voit refuser sa dérogation, il peut, s'il le juge opportun, solliciter le CdP, qui tranchera.
\section{Documents de Référence et applicables}
\subsection{Documents de référence}
\begin{center} \begin{Large}
TODO
\end{Large}  \end{center}
\subsection{Documents Applicables}
TODO
\section{Terminologies et abréviations}

PAQP : Plan d’Assurance Qualité Projet
CdP : Chef de Projet       
RQ : Responsable Qualité       
GEI : Groupe d'Etude Informatique       
MOE : Maitrise d'oeuvre
         
                                                              
\section{Organisation humaine du comité de pilotage du projet}
\subsection{Rôle des différents intervenants ( Intervenants pour la maîtrise d’ouvrage, intervenants pour la maîtrise d’œuvre)}
\subsection{Relations entre les intervenants}
\subsection{Planning des réunions et règles}

\section{Qualité au niveau du Processus}
\subsection{Présentation de la démarche de développement au niveau Projet (ou Système)}
\subsubsection{Généralités              }
\subsubsection{Phase d’Etude Préalable }
\subsubsection{Phase d’Etude Détaillée}
\subsubsection{Phase d’Intégration Système }
\subsubsection{Phase de Validation Système}
\subsubsection{Phase de mise en œuvre sur site pilote}
\subsection{Règles de Qualité pour l’ingénierie concurrente}
\subsubsection{Règles sur la rédaction d’un Cahier des Charges d’un sous-projet}
\subsubsection{Règles sur la définition précise des résultats attendus pour chaque sous-projets}
\subsubsection{Règles sur le suivi qualité des sous-projets}
\subsubsection{Règles sur la définition de critères d’acceptation des sous-projets avant intégration}
\subsection{Présentation des démarche de développement au niveau sous-projets (niv. réalisation)}
\subsubsection{Liste des processus de développement susceptibles d’être retenus pour le développement des sous-projets}
\subsubsection{Description du cycle de développement n°1}
\begin{itemize}
  \item Liste des étapes
  \item Pour étape n°J
  \item Documents en entrée
  \item Documents en sortie
  \item Conditions de validation de l’étape
  \item Suivi de projet
\end{itemize}
 
\section{Documentation (Règles communes au projet)}
\subsection{Structuration de la documentation}
\subsection{Liste des documents de gestion de projet}
\subsection{Liste des documents relatifs à la qualité}
\subsection{Liste des documents techniques et de réalisation}
\subsection{Manuels d’utilisation et de mise en œuvre}

\section{Gestion de configuration (Règles communes au projet)}
\subsection{Conventions d’Identification}
\subsection{Procédures d’identification des éléments de configuration} 
\subsubsection{Responsabilités}
\subsubsection{Procédures de gestion de la configuration (Gestion des Eléments de Configuration Logiciel et des Configurations de Référence)}
\subsection{Gestion des ressources partagées}
\section{Gestions des modifications  (Règles communes au projet)}
\subsection{Origines des modifications}
\subsection{Procédures et organisation des modifications}

\section{Méthodes, Outils et Règles (Règles communes au projet)}
\subsection{Méthodes}
\subsection{Outils (Outils logiciels, autres outils)}
\subsection{Règles et normes   (Documentation, programmation)}		

\section{Contrôle des Fournisseurs}
\subsection{Exigences vis-à-vis des sous-traitants}
\subsection{Exigences vis-à-vis des co-traitants}
\subsection{Logiciels achetés, loués ou imposés}

\section{Reproduction, Protection, Livraison (au niveau projet)}
\subsection{Précautions à prendre lors de la reproduction}
\subsection{Précautions prises pour assurer le stockage des logiciels}
\subsubsection{Protection des données contre les incidents}
\subsubsection{Protection des données contre les agressions extérieures}
\subsubsection{Autres précautions}
\subsection{Modalités de livraison}
\subsubsection{Délais}
\subsubsection{Installation}
\subsubsection{Formation}
\subsubsection{Migration de l’ancien vers le nouveau système}

\section{Suivi de l’application du Plan Qualité}
\subsection{Principes}
\subsection{Interventions du Responsable Qualité (Niv. Projet) sur la démarche de Développement}
\subsection{Modalités de réception des résultats des sous-projets avant intégration	( description par sous-projets)}

\section{Conclusion}



1.3   Objectifs et engagments Qualité du Projet

1.3.1   Objectifs généraux du Projet

Ce projet répond à deux objectifs généraux:

proposer une solution qui répond au plus juste aux besoins exprimés de la MOA
respecter le Système Qualité présenté dans ce document (PAQP), pour assurer la qualité de la solution finale proposée au MOE.
Ce projet est une réponse à l'appel d'offre lancé par le COPEVUE dans le but de proposer un système de monitoring de lieux isolés afin de faciliter la gestion et d'optimiser les processus d'entretiens des réservoirs présents sur les différents sites.

Le système proposé devra répondre aux exigences non-fonctionnelles suivantes, qui ont été développées dans le CdC soumis par le COPEVUE:

Intégration de l'existant
Robustesse
Fiabilité
Evolutivité et Maintenabilité
Limitations Technologiques
Généricité
Réutilisation
Ergonomie
Traçabilité
De plus, les différents coûts engendrés par la solution proposée devront être clairement énoncés. Les catégories de coût concernées sont:

Conception \& Développement du système
Maintenance du système
Fonctionnement du système.
1.3.2   Déclinaison en engagements qualité

TODO: Politique Qualité

1.3.3   Mesure de la Qualité (propriétés et métriques)

TODO

1.4   Conduite de Projet

1.4.1   Relations entre les différents acteurs du projet

TODO: schéma Organigramme des missions assurées au sein du projet (liens hierarchiques et fonctionnels)

1.4.2   Rôles des différents acteurs du projet

Acteur	Responsabilités
COPEVUE (client)	Lanceur de l'appel d'offre "Système de monitoring à distances de sites isolés
MOE (Maîtrise d'Oeuvre)	Il s'agit de l'hexanôme H4314. Il est chargé de répondre à l'appel d'offre de COPEVUE. La MOE est responsable du déroulement du projet et de la solution proposée, tout en tenant compte des contraintes du CdC et des délais fixés par la MOA.
MOA (Maîtrise d'Ouvrage)	La MOA dépend de la COPEVUE. Elle est responsable du CdC, et veille à son respect par la MOE. Elle valide le travail de la MOE.
Comité de Pilotage	Fixe les contraintes et les finalités du projet. Vérifie la politique qualité de la MOE. Analyse, Planifie et décide des actions à entamer. Prévoit des réunions intermédiaires d'avancement de projet.
1.4.3   Rôles des différents acteurs de la MOE

Acteur	Responsabilités
CdP	Dirige le projet, et l'équipe de la MOE à travers la mise en place d'outils de gestion de projet et d'un planning. Il s'occupe d'affecter des ressources à des tâches. Il est le garant du bon déroulement du projet et est le principal interlocuteur pour la communication extérieure à la MOE.
RQ	Met en place et impose une démarche qualité au sein du projet (PAQP, Gestion de de la Documentation, Gestion de la Configuration, Procédures, etc.). Il est le garant de la Qualité.
GEI	Effectue des études et produit de la documentation en fonction des tâches affectées par le CdP. La documentation produite doit respecter les différentes règles du Système Qualité.
1.4.4   Présentation des activités couvertes par le projet

TODO Organigramme des différentes activités qui sont nécessaires au projet.

1.4.5   Relations entre les différents acteurs de la MOE

TODO: schéma

1.4.6   Plannification de projet

La MOA doit clairement indiquer à la MOE les différentes dates importantes du déroulement du projet:

dates de revues intermédiaires
dates de remises des différents livrables
Ainsi, le planning global du déroulement du projet doit être fixé par la MOA dès le début, et doit être le plus rarement possible sujet à modification.

La MOE s'engage à respecter ce planning. Elle est libre d'organiser son propre planning interne de gestion de projet, du moment que les contraintes et les différentes échéanches sont respectées.

Les différentes revues intermédiaires permettront de valider les différents résultats produits par la MOE, d'apporter des critiques, de faire des demandes de modification et éventuellement d'affiner et/ou corriger le CdC de la MOA en fonction des problèmes/questions soulevées

TODO Planning du projet

1.4.7   Suivi du projet

1.4.7.1   Suivi statique
TODO Collecte des données: suivi des tâches, mesure de l'avancement. Réunions de projet (périodicité, ordre du jour, tenue d'un journal de bord, suivi des risques).

1.4.7.2   Suivi dynamique
TODO Mise à jour du planning, adaptation du plan d'action.

1.4.7.3   Suivi prévisionnel
TODO Analyse des dérives, projection de l'avancement sur la suite du projet. Gestion des risques.

1.4.8   Outils de conduite de projet

La plateforme de gestion de projet Redmine sera utilisée.

http://bde.insa-lyon.fr:3000/projects/ingenierie

1.5   Démarche de développement du système d'information

1.5.1   Cycle de développement

TODO Etapes: étude préalable, étude détaillée, étude technique, réalisation, mise en oeuvre (procédure et moyens de reproduction et de vérification par rapport à la référence, droit d'accès, modalités de diffusion).

1.5.2   Description des étapes

TODO - activités, pré-requis - cycle de décision - fournitures attendues (logiciel, documentation...) - méthodes, langages, outils (matériels et logiciels utilisés) - règles et standards applicables (normes ergonomiques ou de programmation, règles de présentation des programmes, ...)

1.6   Gestion de la document

ALREADY DONE!!! voir document gestion de la documentation

1.7   Gestion de la configuration logiciel

1.7.1   Responsabilités

TODO

1.7.2   Identification des éléments

TODO - liste des composants logiciels de l'application, des moyens de développement et de tests - liaison entre les différents éléments

1.7.3   Cycle de vie et états des éléments

TODO - gestion des versions, révisions - vérification, validation

1.7.4   Outils de gestion de configuration

TODO

1.7.5   Production d'état de configuration

TODO

1.7.6   Sauvegarde et archivage

TODO

1.8   Qualité au niveau du Processus de production d'un Système

1.8.1   Généralités

Le développement de ce système se basera sur le "cycle en V", qui produit des livrables à la fin de chacune des phases du cycle.

Ceci permettra de pouvoir valider les livrables produits avant de passer à l'étape suivante en cas de validation, ou alors de recommencer jusqu'à validation dans le cas inverse.

TODO Diagramme du "Cycle en V" du cours.

1.8.2   Détails des différentes phases du Cycle en V

1.8.2.1   Analyse, spécification
Cette phase consiste à étudier de manière précise et détaillée les besoins et faire une étude de l'existant. Une spécification de la cible que doit atteindre le système à développer.

1.8.2.2   Conception Préliminaire
Durant cette phase, il peut être nécessaire de commencer par l'ébauche de plusieurs variantes de solutions et choisir celle qui répond le mieux aux besoins spécifiés lors de la phase précédente tout en tenant compte des contraintes (coûts, etc.) La solution retenue sera ensuite figée, d'où l'importance de cette phase. En parallèle, lors de cette phase, un plan d'intégration et un plan de tests sont élaborés.

1.8.2.3   Conception détaillée
Cette phase sert à détailler ce qui a été donné dans la phase précédente, en composant tout en entités plus élémentaires. Cette décomposition est faite jusqu'à obtenir des entités faciles à tester et à implémenter.

Pendant la conception détaillée, il faut également préparer la vérification des composants logiciels élémentaires du système qui feront l'objet de la phase des tests unitaires.

1.8.2.4   Codage et analyse statique
Lors de cette phase, développement des différents sous-composants listés lors des phases précédentes.

1.8.2.5   Tests unitaires
Phase de tests de manière unitaire de tous les sous-composants de manière indépendante.

1.8.2.6   Intégration
Réception du système par le client et déploiement du nouveau système sur le site.

1.8.2.7   Validation
Vérification et validation de la conformité du système par rapport au CdC par le client.

1.9   Gestion de configuration

1.9.1   Objectifs

La gestion de configuration s'applique à l'ensemble du projet. Elle permet d'assurer la cohérence, et les sauvegardes des différents produits et documents issus du projet.

1.9.2   Outil de gestion de configuration

L'outil de gestion de configuration sera Git avec la plateforme GitHub

http://www.github.com/

1.9.3   Responsabilités

Le RQ sera responsable de la mis en place de l'outil de gestion de configuration, de ses réglages et de sa maintenance.

Les différents membres du projet devront maîtriser l'outil. Pour cela, se référer à la documentation officielle de l'outil.

1.10   Gestion de modification

TODO

1.11   Suivi de l'application du Plan Qualité

1.11.1   Principes

L'application du plan qualité est primordiale si l'on souhaite effectuer un travail de qualité et produire des livrables respectant une certaine homogénéité et cohérence.

L'assurance qualité concerne toutes les procédures qualité établies par le RQ.

1.11.2   Interventions du RQ sur la démarche de développement du projet

Lors des différentes phases de développement du projet, le RQ a pour principales responsabilités: - Le support qualité auprès de l'équipe projet - la validation de la forme des documents produits et livrés selon les règles énoncées dans la Gestion de la Documentation. - la vérification du suivi et de l'application du PAQP par l'équipe projet - la création, le maintien et l'évolution du Système Qualité.

TODO: Suite

1.12   Conclusion

Ce PAQP est un document et un outil qui permet de garantir une solution finale de qualité, à condition qu'il soit bien appliqué.

Il permet également d'assurer que les attentes du client (COPEVUE) vont être prises en compte.

La Qualité a pour vocation d'être toujours améliorée, le présent document peut être sujet à modification.