

\section{Qualité de service : optimisation de l'itinéraire à proposer à la société de maintenance}

    L'architecture complète permettras de connaître précisement en temps et en quantité l'état des cuves à vider ou remplir.
    C'est le moyen de calculer de manière trés efficace un itinaire le plus court possible et le plus efficace, afin de minimiser les consommation et les temps de conduite.
    
    \subsection{Optimisation du trajet}
        
        Le but est de minimiser le trajet, il faudras donc regrouper les lieux à traiter par distance géographique et temporelle.
        La suite logicielle du site centrale récupereras la totalité des informations concernant les sites isolés, c'est donc dans cette suite logicielle qu'il faudra intégrer un module d'optimisation de traitement des sites isolés.
        \\
        Au lieu de soumettre directement les lieux et les dates des sites à traiter à la société de maintenance, il sera possible de soumettre directement un itinéraire précalculé et optimisé.
