\section{STB Non fonctionnelle}

\begin{enumerate}
       \item Fiabilité
       \item Autonomie
       \item Robustesse
       \item Généricité, réutilisation, évolutivité 
       \item Intégration de l'existant
       \item Ergonomie
       \item Traçabilité
       \item Limitations technologiques
 \end{enumerate}

\subsection{Fiabilité}
Comme on souhaite assurer le pilotage, la configuration et la maintenance à distance du système, celuici se doit d’être le plus fiable possible pour limiter au maximum le nombre d’interventions à fournir sur site comme des opérations de maintenance. De même on cherchera à être le plus fiable possible pour tout ce qui concerne le traitement sur site distant ou central, le suivi temps réel ou encore les communications.

\subsection{Autonomie}
	Ce critère apparaît clairement au vu des besoins du nouveau système comme une réelle exigence non fonctionnelle.
En effet les systèmes distant doivent pouvoir fonctionner de façon autonome et sur de longues périodes (voir 1an)
pour réduire au maximum les interventions sur les sites isolés. Cette exigence fonctionnelle est critique pour le bon
fonctionnement de notre système.

\subsection{Robustesse}
Le système se doit donc d’être le plus robuste possible, que ce soit au niveau du traitement sur site central où le système 
doit toujours revenir à un état de fonctionnement stable après la maintenance à distance; où le système doit pouvoir être
 configuré et maintenu à distance sans problèmes. Il doit également pouvoir revenir à un état stable en cas de redémarrage 
intempestif. De plus au niveau du traitement sur site distant le système doit être conçu pour toujours revenir 
à un état stable après un problème type environnemental (EMP) ou autre.

\subsection{Généricité, réutilisation}
Bien que la mise en place de notre système découle d'un besoin spécifique (surveillance à distance de sites 
isolés), notre système doit être adaptable à d'autres besoins avec un minimum de modifications à y apporter. De ce 
fait, durant toutes les phases de spécifications/conception ainsi que de réalisation, la solution qui devra être 
proposée ainsi que ces différentes briques doit être conçus dans un but de réutilisation. En effet notre système 
devra être adaptable pour aboutir à des applications du type : surveillance de trains, surveillance de vieux

\subsection{Intégration de l'existant}
Le système devrait s'adapter aux processus existants qui seront partiellement/totalement conservés.
La reutilisation des processus sera aussi un atout pour le nouveau système.

\subsection{Ergonomie}  
Les interfaces devront être simples, intuivite et adaptées à chaque type d’utilisateur (non-informaticiens). Également, ils doivent être compatible avec
différents plateformes et périphériques (PDA, PC).   

\subsection{Traçabilité} 
Il est indispensable de  pouvoir avoir une trace de toute activité, toute opération effectuée dans la base de données du système. 
Toute intervention (distant ou sur site) devra être répertoriée pour une période de minimum 2 ans. 
Cette trace permettra de d'analyser des données afin de trouver la source éventuelle d'erreur et afin de pemettre l'aide à la décision.


\subsection{Limitations technologiques}
Il existe certaines technologies qui ne sont pas géré entièrement par notre équipe mais mises à disposition par
des société extérieures telle que les réseaux communication. C'est pourquoi il faut s'adapter aux différentes situations rencontrées comme
en cas d'incendie. 
