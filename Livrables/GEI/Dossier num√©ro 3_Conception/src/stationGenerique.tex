\section{Station générique}
    Il s'agit des sites hébergeant le système embarqué ainsi que les balises de location des sites concernées. On pourrait faire une décomposition en sous système pour mieux décrire une station générique. Les sous-systèmes seraient :

       \subsection{Acquisition de données}
       Ce sous-système permet de récolter les données analogiques transmissent par les capteurs grâce aux cartes d'acquisition se trouvant au sein des systèmes embarquées. Ces derniers contiennent également des CAN permettant de transformer ces données, en données numériques exploitables par le sous système "Contrôle". Une fréquence est configuré à l'aide d'une commande à distante du site central, elle définit la périodicité à laquelle le système embarqué doit récupérer les données des capteurs. A ces instants ou le système récolte les données, il scrute chaque capteur et attend une courte durée (timeout) au bout de laquelle, on supposera que le niveau de la cuve concernée n'a sensiblement pas changé, donc qu'il n'est pas pertinent de transmettre une valeur quasiment identique et non critique. Ainsi le système embarquée récoltera uniquement les données pour les capteurs ayant recueillis des changements de niveau de cuve. Ce qui permet d'augmenter la durée de stockage au niveau de la mémoire du système embarqué mais aussi de réduire les flux de données transmis entre les différents systèmes.

       \subsection{Contrôle}
       Ce sous-système permet d'exploiter les données des capteurs à l'aide des micro-contrôleurs des systèmes embarqués et d'effectuer les traitements nécessaires. Il s'agit du cœur des systèmes embarqués. Ainsi après réception de données indiquant par exemple, un niveau bas de la batterie, ou un un niveau critique pour une cuve, des alertes correspondants aux problèmes détectés seront envoyés au système central à travers des protocoles de communication.
       Ce sous-système gère également les communications entre le système et les sources d'énergie, les balises GPS. Il se charge également de sauvegarder temporairement les données des capteurs et de les effacer si nécessaire. C'est également lui qui déclenchera les actions nécéssaires à la réception d'une commande (lors de maintenance à distance).

       \subsection{Communication}
       Ce sous-système permet aux systèmes embarqués de communiquer avec l'extérieur grâce à des protocoles de communication classiques (GPRS en cas de couverture et exceptionnellement par satellite) et ainsi de transmettre les données des capteurs au système central mais aussi de recevoir les commandes lors des opérations de maintenance à distance. Une couche réseau (TCP/IP) du RTOS utilisé permet d'implémenter assez facilement ces communications avec l'extérieur.