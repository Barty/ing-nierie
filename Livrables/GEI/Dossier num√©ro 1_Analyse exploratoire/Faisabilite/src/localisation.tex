\section{Système de localisation}

    \subsection{Système de localisation en temps réel}
        \domaine{GPS}
        { Le système de localisation GPS, d'origin américaine, est actuellement le plus utilisé pour les utilisations civiles.
          Il à une précision de environ 3 mêtres.}
        { Températures de fonctionnement : -30°C à +80°C\\
          Températures de stockage : -40°C à +80°C}
        { \begin{itemize}
                \item Tension d'entrée :
                    \subitem GPS 16x LVS : 3,3 Vdc à 6,0 Vdc régulée, <100 mV ripple
                    \subitem GPS 16x HVS : 8,0 Vdc à 40 Vdc non régulée
                \item Courant d'entrée :
                    \subitem GPS 16x LVS : 90 mA typical
                    \subitem GPS 16x HVS : 100 mA@8 Vdc, 65 mA@12 Vdc, 28 mA@40 Vdc
          \en{itemize}
        }
        { 140€ l'unité.
        }
        {}
        { La couverture en température de fonctionnement et stockage semble plus que correcte.
          La précision suffira largement à la pluspart des usages que l'on pourras en faire.}
	        
	    \subsubsection{Améliorations GPS}
	        Plusieurs systèmes se basent sur le système GPS pour améliorer sa précison de localisation.
	        Cependant, l'amélioration de la précision GPS semble inutile en tenant compte du cahier des charges.

        \subsubsection{Galiléo}
	        Galiléo est le système de navigation par satellite européen prévu pour 2013.
	        
    \subsection{Alternatives}
        Le cahier des charges précise que les appareils de mesures seront utilisé pour vérifier des niveaux dans des cuves ou des entrepot.
        Il n'est peut être pas utile de les localiser en temps réel à tout moment, étant donné qu'ils ne seront pas mobiles.
        On peut penser à effectuer une localisation à l'installation, qui permettras au système embarqué de nous informer de sa position à chaque demande.
        Cela réduit le coût de production, et réduit la consommation d'énergie.
        
        Cependant, pour généraliser au maximum, on peut envisager plusieurs solutions.
