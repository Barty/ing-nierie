\section{RTOS}
	\subsection {Contexe}
		Les RTOS permettront de gérer les communications entre les ressources matérielles et les applications informatiques de notre système. On peut envisager de mettre en place un système d'exploitation bien qu'il existe certainement des solutions couvrant nos besoins. \\~
		
		Les principales contraintes de notre RTOS sont que :
		\begin{itemize}
				\item Il doit être léger car probablement sur micro-contrôleur (espace mémoire limité)
				\item Il doit être le plus stable possible dans un souci d'autonomie
				\item Il doit être générique comme l'ensemble du système
				\item Il doit être fiable : ne pas planter ou se bloquer en fonctionnement car les interventions sur les sites doivent être réduites le plus possible.
		\end{itemize}

		Certains RTOS du marché répondent à nos besoins, le premier cité ci-dessous semble être idéalement conçu pour notre système :
		
		  
\domain{TinyOS}
{RTOS conçu pour des réseaux de capteurs sans fil, son architecture est basée sur une association de composants.}
{Ceci réduit la taille du code nécessaire a sa mise en place (respect contrainte de place en mémoire). La bibliothèque de TinyOS est très complète, on y retrouve des protocoles réseaux, des pilotes de capteurs et des outils d'acquisition de données. L'ensemble de ces composants est adaptable a une application spécifique.}
{TinyOS a été conçu pour réduire au maximum la consommation en énergie d’un capteur. Ainsi, lorsque aucune tâche n’est active, il se met automatiquement en mode veille.}
{Il est open source donc GRATUIT}
{Durant nos recherches exploratoires sur les RTOS du marché, nous avon sremarqué que ce système avait été critiqué pour diverses raisons : mauvaise
gestion de la mémoire (pertes mémoire) et pour des problèmes de synchronisations. Ces doutes seront vérifiées durant l'approfondissement de nos recherches}
{Des informations supplémentaires s'avèrent cruciales}

\domain{LynxOS}
{RTOS de type UNIX conçu pour des systèmes embarqués. Il est conforme au standard POSIX et offre une compatibilité avec Linux}
{Il s'emploie surtout dans des logiciels critiques par exemple : dans l'aviation, le militaire, la fabrication industrielle et dans les communications.
Il est conforme au standard POSIX et offre une compatibilité avec Linux. Il s'emploie surtout dans des logiciels critiques par exemple : dans l'aviation, le militaire, la fabrication industrielle et dans les communications.}
{Sa consommation électrique est très limitée}
{Cout}
{}
{}


Liens utiles : \\~
\begin {itemize}
	\item http://www.tinyos.net/
	\item http://lynxos.org/
\end {itemize}

