\section {Production et gestion de l'energie}
	\subsection {Contexe}
Une part importante du projet consistera en la création de systèmes de gestion et de production d'energie.
Nous pouvons dès à présent définir les diverses sources d'energie nécessaires au bon fonctionnement du système à développer.
Le système général sera tout d'abord composé d'un serveur principal, probablement alimenté par le secteur mais nous pouvons cependant prévoir également un moyen d'alimentation d'urgence en cas de coupure locale du courant.
Les nombreux appareil mobiles présents sur site (systèmes embarqués, capteurs) seront quant à eux munis de batteries.

De nombreuses contraintes apparaissent du fait de l'utilisation d'appareil alimentés par batteries, La durée de vie de ces dernière étant toujours limitée, une bonne gestion de l'energie sera necessaire. Il faudra pour cela concevoir, dans la limite des coût et des moyens technologiques, des système à basse consommation afin d'éviter un surcoût lié à l'utilisation de batteries plus performantes. De nombreuses autres contraintes apparaissent en raison du caractère isolé des sites : les batteries devront être capable de résister à diverses conditions climatiques et présenter d'excellentes performances en matière de fiabilité tout en requérant une maintenance minimale même sur des périodes prolongées de plusieurs années.


	\subsection {Solutions envisagées}
De nombreuses technologies actuelles semblent satisfaire ces contraintes. Les prix et technologies utilisées étant cependant très variés.
La société EnerSage propose effectivement un large choix de batteries pouvant être utilisées dans notre projet : leur gamme de batteries "Power line SC Series" propose en effet des batteries longue durée pouvant alimenter des capteurs à faible consommation énergétique pendant une durée prolongée.
Afin de recharger en continu les batteries présentes sur les sites, nous pouvons installer des équipements d'énergie renouvelable tels que des éoliennes ou des panneaux solaires.
L'entreprise Energie Douce propose plusieurs éoliennes aux caractéristiques largement suffisantes pour notre projet :
\begin{itemize}
	\item Eolienne 12 Volts 200 Watts 465,00 € TTC
	\item Eolienne 12 Volts 400 Watts 699,00 € TTC
	\item Eolienne Air Breeze 24 Volts 300 Watts terrestre 949,00 € TTC
\end{itemize}
Nous pouvons constater que la puissance offerte est largement supérieure à la consommation moyenne d'un système embarqué.

Il en est de même avec les panneaux solaires proposés par la même entreprise :
\begin{itemize}
	\item Panneau solaire haut rendement 12 Volts 130 Watts 549,00 € TTC
	\item Nouveau panneau solaire photovoltaïque haut rendement 24 Volts 185 Watts 699,00 € TTC
	\item Panneau solaire haut rendement 12 volts 80 Watts 336,00 € TTC
\end{itemize}
~\\
Ces solutions permettraient de rendre les sites isolés totalement autonomes en terme d'energie. Une simple maintenance régulière et vérification des batteries permettrait de gérer la ressource énergétique des différents sites.\\


Liens utiles : \\~
\begin {itemize}
	\item http://www.enersafe.fr/
	\item http://www.yuasa.fr/
\end {itemize}
