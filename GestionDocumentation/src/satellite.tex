\section{Introduction}     
  \subsection{Présentation du projet}
    
    \subsubsection{Le contexte}     
     
    \subsubsection{Les objectifs}    
     
  \subsection{Présentation du document}
  
    \paragraph*{\textit{documentation} :}ensemble de documents relatifs à un projet - notice - mode d’emploi - action de sélectionner, classer, utiliser ou diffuser des documents. (Source : Le Petit Larousse - 1994)
    La documentation d’un projet a une importance primordiale : c’est l’outil de communication et de dialogue entre les membres de l’équipe projet et les intervenants extérieurs (membre des instances de pilotage, chef de projet, utilisateurs, etc...). Elle assure aussi la pérennité des informations au sein du projet.
    Afin d’organiser la gestion de la documentation produite par projet, il convient au préalable d’identifier tous les types de documents relatifs aux diverses étapes d’un projet, de les référencer de manière homogène pour ensuite définir un mode de gestion commun à tous les projets.
    \par Le but de ce document est de décrire le fonctionnement de la documentation tout au long du projet.
  
  \subsection{Documents applicables <> Documents de référence}
    \subsubsection{Documents applicables}
      Le dossier ing-nierie/Model/ regroupe l'ensemble des documents applicables. Trois documents y sont présents : \textit{nation.tex} s'occupe de la mise
      en page des documents. Il ne doit en aucun cas être modifié. \textit{main.tex} reprend les informations importantes du document tel que l'auteur,
      la date de création, la version... Son fonctionnement sera expliqué dans le chapitre "Norme de présentation". Enfin \textit{satellite.tex} représente
      le corps du document. Son fonctionnement sera aussi expliqué dans "Norme de présentation".
    \subsubsection{Documents de référence}
      Le site du CNRS : \url{http://www.dsi.cnrs.fr/conduite-projet/}.\\
      Les documents présents à l'adresse \url{/servif-baie/Fichiers_Eleves/Espace Pedagogique/4IF/Developpement de Logiciel/Qualite logiciel/Projet_ingenierie_4IF}.
  \subsection{Terminologie et Abréviations}
  RQ : Responsable Qualité \\
  \LaTeX : \LaTeX est un langage et un système de composition de documents. L'intérêt de \LaTeX est qu'il différencie le fond de la forme.
            \LaTeX exige du rédacteur de se concentrer sur la structure logique de son document, son contenu, tandis que la mise en page du document (césure des mots, alinéas) est laissée au logiciel lors d'une compilation ultérieure.
            Ainsi, la mise en page du document n'est plus à la charge des rédacteurs mais au responsable qualité.
             
\section{Règles générales}
Ce chapitre précise les règles de gestion de la documentation à mettre en oeuvre dans tout projet.
  \subsection{Identification des documents}
  Afin d’assurer l’efficacité de la gestion de la documentation, il faut prévoir une homogénéité d’identification des documents.
Ainsi, chaque document reçoit une référence unique au sein du projet, constituée de plusieurs champs (cette référence apparaît en page de garde et sur chacune des pages du document) :
\textit{nom\_du\_projet/nature\_du\_document/identification\_du\_document}

\par Exemple : "ing-nierie/DossierFaisabilite/Existant" représente la partie "Analyse de l'existant" du dossier de Faisabilité.

  \subsection{Norme de présentation}
    \subsubsection{Structure}
      Du fait de l'utilisation de \LaTeX, un document ne correspond pas à un fichier mais à un dossier (portant le nom du document), composé de deux dossiers :
       "/bin" : contient le document final (après compilation) en pdf.
       "/src" : contient plusieurs fichiers .tex (au minimum 3) :
          \paragraph{nation.tex} : ne doit en aucun cas être modifié. Il s'occupe de la mise en page du document.\\
          
          \paragraph{main.tex} : reprend les informations importantes du documents pour la mise en page. \\
\begin{center}
\begin{boxedverbatim}
  \documentclass[a4paper, 11pt, titlepage]{article}
  \usepackage[english, french]{babel}
\usepackage[utf8]{inputenc}

\usepackage{graphicx}
\usepackage{fancyhdr}
\usepackage{lastpage}
\usepackage{amsmath}
\usepackage{xspace}
\usepackage{textcomp}

\usepackage{hyperref}

\usepackage[top=20mm, bottom=20mm, left=25mm, right=25mm]{geometry}

\pagestyle{fancy}

\usepackage{helvet}
\usepackage{bbm}

\usepackage{verbatim}
\usepackage{amsmath}
\usepackage[table]{xcolor}
\definecolor{bleugris}{rgb}{.2,.4,.5}

\definecolor{colKeys}{rgb}{0,0,1} 
\definecolor{colIdentifier}{rgb}{0,0,0} 
\definecolor{colComments}{rgb}{0,0.5,1} 
\definecolor{colString}{rgb}{0.6,0.1,0.1} 

\usepackage{listings}

% Permet l'ajout de code par insertion du fichier le contenant
% Les arguments sont :
% $1 : nom du fichier à inclure
% $2 : le type de langage (C++, C, Java ...)
\newcommand{\addCode}[2]{%

  % Configuration de la coloration syntaxique du code
  \definecolor{colKeys}{rgb}{0,0,1}
  \definecolor{colIdentifier}{rgb}{0,0,0}
  \definecolor{colComments}{rgb}{0,0.5,1}
  \definecolor{colString}{rgb}{0.6,0.1,0.1}

  % Configuration des options 
  \lstset{%
    language = #2,%
    identifierstyle=\color{colIdentifier},%
    basicstyle=\ttfamily\scriptsize, %
    keywordstyle=\color{colKeys},%
    stringstyle=\color{colString},%
    commentstyle=\color{colComments},%
    columns = flexible,%
    %tabsize = 8,%
    showspaces = false,%
    numbers = left, numberstyle=\tiny,%
    frame = single,frameround=tttt,%
    breaklines = true, breakautoindent = true,%
    captionpos = b,%
    xrightmargin=10mm, xleftmargin = 15mm, framexleftmargin = 7mm,%
  }%
    \begin{center}
    \lstinputlisting{#1}
    \end{center}
}

\newcommand{\nTitle}[1]{%
	\clearpage
	\vspace*{\fill}		%
	\begin{center}	%
		\part{#1}		%
	\end{center}
	\vspace*{\fill}		%
	\clearpage
}

\newenvironment{nAbstract} 		%
{ 								%
	\newpage 					% 
	\vspace*{\fill}				%
	\begin{center}			 	%
		\begin{abstract}		%
}{								%
		\end{abstract}			%
	\end{center}				%
	\vspace*{\fill}				%
	\newpage					%
}


\newcommand{\nClass}[1]{{\color{bleugris}{\textsl{\textbf{#1}}}}}
\newcommand{\nParameter}[1]{{\color{gray}{\textbf{#1}}}}
\newcommand{\nMethod}[1]{{\color{gray}{\textbf{#1}}}}
\newcommand{\nConstant}[1]{\texttt{\uppercase{#1}}}
\newcommand{\nKeyword}[1]{\textsl{\textbf{#1}}}

\graphicspath{{../SourcesMatlab/}}

% Conversion nombres arabes / romain
\makeatletter
\newcommand{\rmnum}[1]{\romannumeral #1}
\newcommand{\Rmnum}[1]{\expandafter\@slowromancap\romannumeral #1@}
\makeatother

\setlength{\headheight}{14pt}

\fancyhf{}


\makeatletter
\def\clap#1{\hbox to 0pt{\hss #1\hss}}%
\def\ligne#1{%
\hbox to \hsize{%
\vbox{\centering #1}}}%
\def\haut#1#2#3{%
\hbox to \hsize{%
\rlap{\vtop{\raggedright #1}}%
\hss
\clap{\vtop{\centering #2}}%
\hss
\llap{\vtop{\raggedleft #3}}}}%
\def\bas#1#2#3{%
\hbox to \hsize{%
\rlap{\vbox{\raggedright #1}}%
\hss
\clap{\vbox{\centering #2}}%
\hss
\llap{\vbox{\raggedleft #3}}}}%
\def\maketitle{%
\thispagestyle{empty}\vbox to \vsize{%
\vfill
\vspace{1cm}
\begin{flushleft}
\usefont{OT1}{ptm}{m}{n}
\huge \@title
\end{flushleft}
\par
\hrule height 4pt
\par
\begin{flushright}
\usefont{OT1}{phv}{m}{n}
\Large \@author
\par
\end{flushright}
\vspace{1cm}
\vfill
\vfill
\bas{}{\@blurb \vspace{1cm}}{}
}%
\cleardoublepage
}
\def\date#1{\def\@date{#1}}
\def\author#1{\def\@author{#1}}
\def\title#1{\def\@title{#1}}
\def\blurb#1{\def\@blurb{#1}}
\author{}
\title{}
\blurb{}
\makeatother

  \title{_________________\\~\\ \small{Version ____________}}
  \blurb{
  \begin{tabular}{|p{\linewidth}|}
  Rédacteur(s) : ______________	\\ \hline
  Date de création : __________________\\
  Date de modification : \today \\ \hline
  Etat (En cours/à valider/à corriger/validé) : ______________\\ \hline
  Responsable qualité : Baptiste Lecornu\\ \hline                     
  \end{tabular}
  }
  \author{H4212}
  \lhead{Hexanome 4212 - Projet Ingénierie}
  \rhead{\includegraphics [width=1.5cm]{insa-couleur.jpg}}
  \rfoot{\thepage\ de \pageref{LastPage}}
                        
  \begin{document}
  \maketitle
  \newpage
  \tableofcontents 
            
  \input{__Satellite1.tex__}
  ...
  \input{__SatelliteN.tex__}
            
  \end{document}
\end{boxedverbatim}
\end{center}
\par Les espaces vides soulignés doivent être renseignés par les rédacteurs, à savoir le nom du document; sa version; le nom des rédacteurs,
la date de création et l'état. Pour l'état, les rédacteurs ne peuvent mettre que "En cours" ou "A valider". "A corriger" ou "Validé" ne peuvent
être attribués que par le responsable qualité. Enfin, la liste de "SatelliteN.tex"(explication à la suite) doit être mise à jour à chaque modification.\\
          \paragraph{satelliteN.tex} : correspond au corps du texte. Peut être divisé en plusieurs fichiers .tex, notamment lorsque le document
                est écrit par plusieurs rédacteurs qui s'occupent chacun d'une partie différente. Les fichiers portent alors le nom de cette partie
                est doit être inclus dans main.tex de la manière suivante :
\begin{center}\begin{boxedverbatim}
\input{Nom_du_Fichier.tex}
\end{boxedverbatim} 
\end{center}
           
       
      
      \paragraph{Récapitulatif :\\~\\}  
       Au final, doit être présents pour chaque document :
      \begin{itemize}
        \item le titre du document,
        \item la référence du document,
        \item la date de dernière mise à jour,
        \item le numéro de version de l'application concernée par le document (VX.x),
        \item le nom de l’auteur (ou des auteurs),
        \item pour les documents faisant l'objet d'une vérification et/ou d'une validation, le cartouche de visa (noms des destinataires, objet de la diffusion, dates de visa).\\
      \end{itemize}                                                                                                        
      D’autre part, sur chaque page du document préciser : \\
      \begin{itemize}
        \item le titre du document,
        \item la référence,
        \item le logo de l'entreprise,
        \item le numéro de page / nombre de pages total.
      \end{itemize}         
      
      Chaque document comprend un sommaire, qui reprend les titres des chapitres et des différents paragraphes et précise les numéros de pages correspondants.
  
   
  \subsection{Etats d'un documents}
  
    Quatre états différents permettent de définir un document :
    \begin{itemize}
      \item \textit{en cours} : état initial de tout document, il signifie que le document n'est pas encore terminé.
      \item \textit{à valider} : état de transition signifiant que la rédaction du document est terminée et qu'il doit donc être vérifié par le RQ.
      \item \textit{à corriger} : état \textbf{donné par le RQ} signifiant que le document contient des erreurs de cohérence et qu'il doit être modifier en conséquence.
      \item \textit{validé} : état final signifiant que le document est apte à être consulté et/ou communiqué.
    \end{itemize}
  
  \subsection{Cycle de vie d'un document}
    \subsubsection{Production du document}
      La production du document se fait en \LaTeX au sein des fichiers "SatelliteN.tex".\\ Aucune mise en garde sur la mise en page n'est à
      prescrit, puisque celle-ci se fait de manière automatique.\\ A la fin de chaque phase de travail, le rédacteur se doit de compiler ça partie
      afin de corriger toute erreur éventuelle de programmation.
    \subsubsection{Vérification/Validation du document}
    
  \subsection{Gestion des versions}
  
\section{Gestion de la documentation produite}
  \subsection{Gestion physique des fichiers contenant les documents}     
    \subsubsection{Répertoires}
    \subsubsection{Noms des fichiers}
    
    
\section{Annexes}
  \subsection{Différence entre un livrable intermédiaire et une ébauche}     
    \subsubsection{Document de type "LIVRABLE INTERMEDIAIRE" bien identifié dans le projet}
    \subsubsection{Document de type "DRAFT"}
  
  \subsection{Plans types}     
    \subsubsection{Plan type d'un Dossier de Synthèse}
    
    \subsubsection{Plan type du dossier n°1 "Etude de faisabilité"}
      \paragraph*{\\Objectif :}
        présenter de manière succinte l'existant, ses points forts et ses faiblesses. Il doit
        présenter une étude de faisabilité par rapport aux technologies émergentes et fiables, et donner quelques pistes d'évolution.
      \paragraph*{Plan type :\\}
        \begin{enumerate}
          \item Introduction
          \item Analyse de l'existant
            \begin{enumerate}
              \item Analyse du métier
              \item Analyse des savoir-faire et des processus
              \item Analyse du matériel utilisé
            \end{enumerate}
          \item Etude de faisabilité
            \begin{enumerate}
              \item Synthèse sur Système embarqué
              \item Synthèse sur Gestion de l'énergie
              \item Synthèse sur Capteurs
              \item Synthèse sur Systèmes de communication
              \item Synthèse sur Systèmes de localisation
              \item ...
            \end{enumerate} 
          \item Conclusions 
        \end{enumerate}
        
    \subsubsection{Plan type du dossier n°2 "Spécification Technique des Besoins du Système"}
    \subsubsection{Plan type du dossier n°3 "Conception du nouveau système"}
    \subsubsection{Plan type d'un Plan d'Assurance Qualité Projet (PAQP)}
    \subsubsection{Plan type d'un Plan de Management de Projet (PMP)}