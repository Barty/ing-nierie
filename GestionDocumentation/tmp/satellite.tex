\section{Introduction}     
  \subsection{Présentation du projet}  
    \subsubsection{Le contexte}      
    \subsubsection{Les objectifs}     
  \subsection{Présentation du document}
  \paragraph*{\textit{documentation} :}ensemble de documents relatifs à un projet - notice - mode d’emploi - action de sélectionner, classer, utiliser ou diffuser des documents. (Source : Le Petit Larousse - 1994)
La documentation d’un projet a une importance primordiale : c’est l’outil de communication et de dialogue entre les membres de l’équipe projet et les intervenants extérieurs (membre des instances de pilotage, chef de projet, utilisateurs, etc...). Elle assure aussi la pérennité des informations au sein du projet.
Afin d’organiser la gestion de la documentation produite par projet, il convient au préalable d’identifier tous les types de documents relatifs aux diverses étapes d’un projet, de les référencer de manière homogène pour ensuite définir un mode de gestion commun à tous les projets.
\par Le but de ce document est de décrire le fonctionnement de la documentation tout au long du projet.
  \subsection{Documents applicables Documents de référence}
    \subsubsection{Documents applicables}
      DGA - RG Aéro 00040 : Document cité directement ou indirectement dans un contrat ou dans un autre document et requis contractuellement comme devant être impérativement appliqué au titre de ce contrat ou de cet autre document.
    \subsubsection{Documents de référence}
      DGA - RG Aéro 00040 : Document cité dans un contrat ou dans un autre document et pouvant être utilement consulté pour l'exercice des activités liées au contrat.
  \subsection{Terminologie et Abréviations}
\section{Règles générales}
  \subsection{Identification des documents}
  \subsection{Norme de présentation}
  \subsection{Etats d'un documents}
  \subsection{Cycle de vie d'un document}
    \subsubsection{Production du document}
    \subsubsection{Vérification/Validation du document}
    \subsubsection{Archivage du document}
  \subsection{Gestion des versions}
\section{Gestion de la documentation produite}
  \subsection{Classement de la documentation}
  \subsection{Gestion physique des fichiers contenant les documents}     
    \subsubsection{Répertoires}
    \subsubsection{Noms des fichiers}
    \subsubsection{Procédures de sauvegarde et archivage}
\section{Gestion de la documentation papier}
  \subsection{Recherche de documents papiers}
  \subsection{Gestion des emprunts}
\section{Communication interne}
\section{Annexes}
  \subsection{Modèles de documents}
  \subsection{Différence entre un livrable intermédiaire et une ébauche}     
    \subsubsection{Document de type "LIVRABLE INTERMEDIAIRE" bien identifié dans le projet}
    \subsubsection{Document de type "DRAFT"}
  \subsection{Plans types}     
    \subsubsection{Plan type d'un Dossier de Synthèse}
    
    \subsubsection{Plan type du dossier n°1 "Etude de faisabilité"}
      \paragraph*{\\Objectif :}
        présenter de manière succinte l'existant, ses points forts et ses faiblesses. Il doit
        présenter une étude de faisabilité par rapport aux technologies émergentes et fiables, et donner quelques pistes d'évolution.
      \paragraph*{Plan type :\\}
        \begin{enumerate}
          \item Introduction
          \item Analyse de l'existant
            \begin{enumerate}
              \item Analyse du métier
              \item Analyse des savoir-faire et des processus
              \item Analyse du matériel utilisé
            \end{enumerate}
          \item Etude de faisabilité
            \begin{enumerate}
              \item Synthèse sur Système embarqué
              \item Synthèse sur Gestion de l'énergie
              \item Synthèse sur Capteurs
              \item Synthèse sur Systèmes de communication
              \item Synthèse sur Systèmes de localisation
              \item ...
            \end{enumerate} 
          \item Conclusions 
        \end{enumerate}
        
    \subsubsection{Plan type du dossier n°2 "Spécification Technique des Besoins du Système"}
    \subsubsection{Plan type du dossier n°3 "Conception du nouveau système"}
    \subsubsection{Plan type d'un Plan d'Assurance Qualité Projet (PAQP)}
    \subsubsection{Plan type d'un Plan de Management de Projet (PMP)}