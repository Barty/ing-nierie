\section{Introduction}
  \subsection{Contexte global}
    Le COPEVUE (Comité pour la Protection de l'EnVironnement de l'UE), présidé par le commissaire Norvégien souhaite étudier \textbf {un système
    de monitoring de sites isolés} .
    De nombreuses régions de l'UE, se situant dans les pays Nordiques, ou certaines régions méditérranéennes (à haut risque en terme d'incendies) sont peu peuplées et peu aisément accessibles. Néanmoins
    de nombreux lieux de travail existent dans ces régions tels que ceux nécessaires à l'abattage de bois, l'installation de réseaux (éléctrique ou de télécommunication),
    de stations de pompage ou encore des lieux dédiés à certaines études sur la faune et la flore. Ces lieux sont bien souvent isolés et disséminés loin des villes et des grans centres et doivent donc être autonomes en termes 
    d'énergie, de déchets etc...
    Les réservoirs doivent être surveillés pour être ravitaillés, nettoyés ou vidés avant que le niveau n'atteigne un seuil critique. 
    Actuellement la surveillance de ces lieux est assurée par le propriétaire du lieu de travail qui, en fonction du niveau qu'il constate, 7
    avertit la société chargée de s'occuper du réservoir pour qu'elle vienne le remplir / vider.
    Les sociétés chargées de la maintenance des réservoirs doivent donc équiper et envoyer un camion pour s'occuper des réservoirs lorsque 
    le propriétaire en fait la demande. Mais lorsque ces sociétés s'occupent de plusieurs dizaines de sites différents.
    Par ailleurs, pour des raisons de coût, on a pu constater des manquements vis-à-vis des exigences de surveillance qui peuvent se traduire par des risques 
    de désastres stratégiques dans des forets méditéranéennes ayant de fort risque d'incendie.\\
    La surveillance du niveau de ces réservoirs doit donc être assurée d'une autre manière afin de permettre aux sociétés chargées de leur maintenance de planifier les trajets des camions
    afin de faire des économies logistiques tout en garantissant certaines autres exigences.
  \subsection{Objectif du document}
    Le but principal de ce document est de synthètiser l'étude technique réalisée par notre équipe en présentant de manière claire et détaillée une vision globale du futur système.\\
    Ce document s'adresse aussi bien à COPEVUE qu'aux menbres de l'équipes pour les aider à rédiger tout autre document
  \subsection{Références}
    \begin{itemize}
      \item Dossier d'initialisation
      \item Dossier Spécifications Techniques des Besoins
      \item Dossier de Conception du Système
    \end{itemize}  
      
\section{L'existant}
  \subsection{Cadre de l'étude et objectifs}  
  systeme de monitoring
  Une composante importante : le caractère hostile de l'environnement hostile
  \begin{itemize}
    \item Fiabilité du matériel
    \item Généricité du système
    \item Robustesse logicielle
  \end{itemize}
  \subsection{Critique de l'existant} 
    

\section{Solution proposée}  
  \subsection{Architecture}
  \subsection{Fonctionnement général}                          
    \subsubsection{Gestion de l'énergie}
    \subsubsection{Récupération de l'information}
    \subsubsection{Communications Capteurs - Site local}
    \subsubsection{Communications Site locaux - Station central}
    \subsubsection{Exploitation des données}
    \subsubsection{Maintenance}
    \subsubsection{Aide à la décision et communication externe}
\section{Apport de la solution / Thèmes de progrès}
\section{Investissement et mise en place de la solution}  
  \subsection{Coûts estimés}
  \subsection{Mise en place du système}
\section{Conclusion}  