\section{Introduction}
TODO
\section{Contexte du document}
TODO
\section{Rappel du problème}
\subsection{Le contexte}
TODO
\subsection{Les objectifs}
TODO
\section{Les contraites générales}
\subsection{Confidentialités}
TODO
\subsection{Existant}
TODO
\subsection{Exigences non fonctionnelles}
TODO
\section{Organisation du travail}
\subsection{Ressources humaines}
\subsubsection{Chef de projet et coordinateur}
BACHATENE Chafik
TODO : rôle du chef de projet
\subsubsection{Responsable qualité / méthode / documentation}
LECORNU Baptiste
TODO : rôle du responsable qualité
\subsubsection{Groupe d'Etude Informatique}
\begin{itemize}
\item BROCHOT Adrien
\item BRODU Etienne
\item DIAKITE Naby Daouda 
\item PHAN Duc Than
\end{itemize}
TODO : Rôles des GEI

\subsection{Activités}
\subsubsection{Production}
\begin{itemize}
\item Etude, expertise (etat de l'art, faisabilité ...)
\item Analyse des besoins techniques
\item Conception
\end{itemize}
\subsubsection{Controle (qualité)}
Voir RQ
\subsubsection{Gestion de projet}
\begin{itemize}
\item Suivi et contrôle
\item Validation
\item Réunion
\item ...
\end{itemize}
\section{Liste des livrables attendus}
\subsection{Livrables de production}
\begin{itemize}
\item Première phase
\begin{enumerate}
\item L'analyse exploratoire fournit les livrables suivants :
\begin{itemize}
\item Synthèse de l'analyse des besoins (description)
\item Synthèse de la faisabilité (description)
\item Synthèse des axes d'amélioration (relevé de décisions)
\end{itemize}
\item L'analyse des besoins techniques doit fournir \textbf{le cahier des charges du prochain système} 
\begin{itemize}       
\item Synthèse des besoins fonctionnels
\item Synthèse des besoins non fonctionnels       
\end{itemize}       
       
\item La conception du nouveau systeme doit fournir le dossier de conception 
\begin{itemize}
	   
\item Description des composants du système
\item Draft de la description de la station générique
\item Draft de la description du site central
\item Draft de la description des bords nécessaires
\item Draft de l'architecture applicative   
\begin{itemize}

\item Draft de la description des données manipulées
\item Description choix modélisation      
\item Conception Modèle des application
\item Liste des applications nécessitant une IHM          
\end{itemize}          
\item Draft protocoles utilisés
\item Draft analyse de la complexité
\end{itemize}       
\end{enumerate} 
\item Deuxième phase
\begin{enumerate}
\item Décomposition du système en sous-systèmes 
\begin{itemize}
\item Liste des sous-systèmes 
\item Draft des differents cahiers des charges
\end{itemize}     
     

\item Spécification des interfaces
\begin{itemize}    
\item Spécification détaillée des sens de communication
\item Spécification détaillée des protocoles de communication
\item Spécification détaillée des données échangées
\item Elaboration d'un diagramme de collaboration
\end{itemize}        

\item Rédaction des cahiers des charges informatique
\begin{itemize}
\item Rédaction cahier des charges 
\end{itemize}
\end{enumerate}


\end{itemize}
\subsection{Livrables de suivi de projet}
\begin{itemize}
\item Dossier d'initialisation
\item Fiche commerciale
\item PMP
\item Procédure d'aide à la rédaction d'un cahier des charges
\item Bilan
\end{itemize}
\subsection{Livrables de qualité}
Voir le responsable qualité
\section{Organigramme des tâches}
\subsection{Macro-Phasage}
\begin{enumerate}
\item Tâches de production :
    Par livrable et par phase 
    \begin{itemize}
    \item Pase 1:
    \begin{enumerate}
    \item Synthèse de l'analyse des besoins 
    \begin{itemize}
			\item Rassembler les questions et réponses de la première séance et les partager
			\item Lire les documents et bien comprendre
			\item Lire le cahier des charges 
			\item Synthétiser les besoins en suivant le formalisme défini par le RQ
			\end{itemize}
	\item Synthèse de la faisabilité :
	\begin{itemize}
			\item Trouver les sources pour analyser les faisabilités
			\item Synthétiser les faisabilités par domaines (formalisme défini par le RQ)
			
			\end{itemize}
	\item Axes d'amélioration :
	\begin{itemize}
		    \item Dégager un ensemble d'axes d'amélioration (début du marketing)
	        \item Synthèse des besoins fonctionnels 	       
	        \item Description et détail des besoins fonctionnels des aspects suivants
	        \begin{itemize}
	            \item Les besoins d'interfaces
	            \item Les besoins de communication
	            \item Les besoins de stockage
	        \end{itemize}
	        \item Explication de l'évolutivité des besoins
	        \begin{itemize}
	            \item Citer les besoins pouvant s'ajouter
	            \item Citer les méthodes de gestion et d'integration de nouveaux besoins
	        \end{itemize}
	        
	    	\item Synthèse des besoins non fonctionnels
	    	\begin{itemize}
				\item Faire une liste (exhaustive) des besoins non fonctionnels
				\item Faire un tableau des besoins par priorité
			\end{itemize}
	\end{itemize}
	    	
	\item D3 Voir livrables
	\end{enumerate}
	\item Phase 2 :	
	\item D4 Voir livrables
	\item D5 Voir livrables
	\item D6 Voir livrables
    
    \end{itemize}
    
\item Tâches de suivi (gestion de projet)
	\begin{itemize}
	\item Rédaction dossier d'initialisation
	\item Répartition des rôles
	\item Rédaction dossier de suivi
	
	\end{itemize}
	
\item Tâches de controle qualité :
\begin{itemize}
    \item Validation des documents (livrables
	\item Formation outils de travail
\end{itemize}	
\end{enumerate}
\subsection{Diagramme de Gantt}
Voir diagramme de gantt Redmine (saise des tâches en cours...)
\section{Modalités de suivi}
\subsection{Les règles de suivi}
\subsection{Les outils utilisés}
TODO : Redmine, expliquer brièvement le principe
TODO : Compte rendu réunion, principe 
TODO : Lien avec le responsable qualité
\subsection{les procédures de révision du planning}
\section{Gestion des risques}
\subsection{Risques concernant l'application du projet}
\subsection{Risques propres au projet}
TODO : Tableau des risques et solutions


