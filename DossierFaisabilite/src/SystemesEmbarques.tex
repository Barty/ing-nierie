\documentclass{article}

\begin{document}


\subsection{Systèmes embarqués}

		Le but du système embarque est de récolter les informations des capteurs (données sur le site, la localisation..),
	puis de les transmettre en temps réel au système central.
		
		Les principales contraintes du système embarqué sont la fiabilité, la robustesse car le système doit nécessiter peu d'interventions sur site et doit fonctionner au moins 2ans. 
		
		Cette dernière contrainte implique aussi que le système doit avoir une très faible consommation énergétique. Il doit également être configurable a distance (du site central).
		
		Concernant l'interface utilisateur exigée, peu de contraintes, elle doit être très basique, les opérations de maintenance sur site seront rares et ne nécessiteront pas une interface utilisateur complexe proposant énormément de fonctionnalités.
		
		Les systèmes embarques vont utiliser une de ces solutions :
		\begin{itemize}
				\item Des micro-processeurs a faible consommation énergétique
				\item Des micro-contrôleurs, dont la partie logicielle est en partie ou entièrement programmée dans le matériel, généralement en mémoire morte (ex : ROM).
		\end{itemize}

		Certains systèmes existent aujourd'hui et permettent de couvrir les besoins que nous venons d'exprimer, un d'entre eux est : Watchman500 Node. 

\domaine{Nom du domaine}
{ Il s'agit d'un système intelligent, configurable, extensible, et prenant en E/S des capteurs avec une plateforme de communication bidirectionnelle.}
{Il est conçu pour être déployer dans des environnement divers (mêmes ceux contraignant et nécessitant une grande robustesse..). Il est intégrable avec la plupart des capteurs du marche et différents logiciels. Bien entendu la société AXYS Technologies INC propose des capteurs idéalement conçus pour Watchman500 Node ainsi qu'un système de pilotage, monitoring a distance et bien entendu de collectes de l'ensemble des informations récoltées.}
{On peut réduire la vitesse du processeur, le mettre en veille ce qui permet de réduire considérablement sa consommation. Voici certains valeurs indicatives : 
\begin{itemize}
       \item Exploitation normale : 18V, <35mA
       \item Modes de puissance faible disponibles
       \item 142μA en mode veille
\end{itemize}
}
{}
{Lien utile : http://www.axystechnologies.com/BusinessUnit/CoreTechnology/WatchMan500Node/tabid/102/Default.aspx}
{Ce système semble idéal pour couvrir nos besoins.}

		

\end{document}
