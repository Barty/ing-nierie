\section {Capteurs}

	\subsection {Contexte} 
	L’objectif est de surveiller en temps réel des réservoirs stockant divers produits et se situant dans un site isolé. Les capteurs doivent être capables de détecter si le niveau du contenu des réservoirs a dépassé un certain seuil pré-défini pour ensuite envoyer cette information au système embarqué. \\~
Pour répondre à ces besoins, plusieurs solutions existent : 

	\subsection {Mesure de niveau continue}
 	Le capteur et son conditionneur délivrent un signal proportionnel au niveau de liquide dans le réservoir. À chaque instant, l'opérateur connaît exactement le volume du liquide (ou le volume encore disponible dans le réservoir).  On peut trouver dans cette catégorie deux types de capteur  courants : 
	\begin {description}
		\item [Le capteur à flotteur]
		Il se maintient à la surface du liquide dans les réservoirs de stockage à distance et mobiles. 
		\begin {itemize}
			\item Avantages
			\begin {itemize}
				\item Il s'agit d'une mesure directe de la hauteur du liquide et elle ne dépend pas de sa masse volumique. 
		   		\item Sur une grande étendue de mesure (plusieurs mètres), la mesure est précise.
		   		\item Simple et économique.
			\end {itemize}
			
			\item Inconvénients
			\begin {itemize}
				\item Le flotteur est en contact direct avec le liquide : les produits corrosifs sont donc à exclure
				\item La mesure est très sensibles aux perturbations à la surface du liquide (vague, remous,...) 
				\item Il est nécessaire d'entretenir régulièrement le système
			\end {itemize}
		\end {itemize}
		\item [Mesure par plongeur]
			Un cylindre est immergé verticalement dans le liquide contenu dans le reservoir. La hauteur de ce plongeur doit au moins être égale à la 				hauteur maximale du liquide dans le réservoir.Le plongeur est suspendu à un capteur dynamométrique.
		\begin {itemize}
			\item Avantages
			\begin {itemize}
				\item Bonne precision
				\item La mesure n'est pas influencée par les modifications de surface du liquide (mousses,...)
 			\end {itemize}

			\item Inconvénients
			\begin {itemize}
				\item Le plongeur est en mouvement et en contact avec le liquide S'il y a des dépôts sur le plongeur, cela fausse la mesure. 
				\item La mesure est modifiée par les mouvements du liquide. 
				\item La mesure n'est valable que pour les liquides 
				\item Le coût d'achat et d'entretien est important.
			\end {itemize}
		\end {itemize}
	\end {description}

	\subsection {Mesure de niveau continue}
	La détection de niveau est une mesure binaire, c'est-à-dire que le capteur délivre une information binaire indiquant si le niveau seuil défini est atteint ou 		pas. La détection de niveau ne permet donc pas de connaître le volume de liquide contenu dans le réservoir, mais permet de savoir si le liquide a atteint un 		seuil. En général, cette solution propose des capteurs plus simples et moins couteux que ceux de la 1er catégorie.

	\begin {description}
		\item [Exemple :]
	Détection par micro-ondes: base sur l’électromagnétique. Lorsque le produit s'interpose entre l'émetteur et le récepteur, le signal reçu par le récepteur est atténué,  l'état de la valeur binaire de sortie.
	\end {description}
	\begin {itemize}
		\item Avantages 
		\begin {itemize}
			\item Sans contact. 
			\item Quasiment aucun entretien n'est nécessaire. 
			\item Cette méthode convient aussi bien pour les liquides que les solides…
		\end {itemize}
		\item Inconvénients
		\begin {itemize}
			\item Ne convient donc pas pour les produits plastiques. (tranverse)
		\end {itemize}
	\end {itemize}

Le capteur pourrait être doté par un générateur d'énergie (cellule solaire) pour générer l'énergie lui-même. Des technologies telles que celle proposées par ENOCEAN proposent des capteurs sans dépense d'energie, nous permettant de nous affranchir du problème d'alimentation des capteurs.


Liens utiles : \\~
\begin {itemize}
	\item http://www.endress.com/fr/detection-mesure-niveau.html
	\item http://www.directindustry.fr/
	\item http://www.enocean.com/en/
\end {itemize}
