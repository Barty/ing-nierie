\section{Aide à la rédaction du PAQL du sous-système "Système Embarqué"}

\subsection{Etude des exigences non fonctionnelles du CdC du logiciel}
Dans cette partie nous définissons les facteurs selon l'usage du logiciel. Ceux-ci sont classés selon un ordre de priorité, qu'il faudra respecter avec précision(certains facteurs étant en opposition, la priorité permet de savoir où se placer). Ces facteurs sont défini à partir du cahier des charges du sous-systèmes "Système Embarqué", notamment grâce aux exigences non-fonctionnelles.

\subsubsection{Fiabilité} 
On demande au logiciel d'avoir le minimum de panne afin de répondre au besoin de surveillance continue des risques environnementaux. On veut une sûreté de fonctionnement, une stabilité dans le temps.

\subsubsection{Maintenabilité / Durabilité}
Au vu de son caractère de surveillance, le logiciel se doit d'être non seulement fiable mais aussi facilement réparable. La localisation et la correction d'erreurs doit s'effectuées dans les plus brefs délais. L'objectif est aussi d'assurer une longue durée de vie au logiciel.

\subsubsection{Efficacité} 
Enfin, le dernier facteur à prendre en compte est le temps d'éxecution. La réponse face à un problème au niveau des capteurs se doit d'être la plus rapide possible.
Ce dernier facteur va à l'encontre du principe de maintenabilité. L'identification des priorités de ces facteurs permet de dire qu'il faudra privilégier la maintenabilité à l'efficacité sans pour autant mettre l'efficacité de côté.


\subsection{Identification des exigences de qualités et classification}