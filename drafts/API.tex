\section{API}

\subsection{Communication entre les sociétés de maintenance et le système central}
\begin{itemize}
	\item List$<Planning>$ consulterPlannings()
	\begin{description} 
		\item[Rôle :] Permet à la société de maintenance de récupérer les nouvelles données présentes sur le serveur.
		\item[Contrat :] AUCUN
		\item[En entrée :] AUCUN
		\item[En sortie :] La liste des nouveaux plannings disponibles sur le site central.
	\end{description}
	~\\
	\item mettreAJourPlanning(Planning p)
	\begin{description} 
		\item[Rôle :] Permet à la société de maintenance de mettre à jour le planning (opérations sur site terminée, opérations inutiles...)
		\item[Contrat :] AUCUN
		\item[En entrée :] le planning concerné 
		\item[En sortie :] Aucune sortie
	\end{description}
	~\\
	\item AdresseGPS coordonneesGPS(Site s)
	\begin{description} 
		\item[Rôle :] Permet de récupérer les coordonnées GPS d'un site passé en paramètre
		\item[Contrat :] Le site doit exister
		\item[En entrée :] un site générique
		\item[En sortie :] les coordonnées GPS de ce site
	\end{description}
	~\\

\end{itemize}

\subsection{Communication entre les sites génériques et le système central}
\begin{itemize}
	\item List$<RapportSite>$ MiseAJourEtatSite(List$<Site>$  s) 
	\begin{description} 
		\item[Rôle :] Permet de rafraichir l'état des capteurs des sites passés en paramètre et de récupérer les données actuelles instantanément.
		\item[Contrat :] AUCUN 
		\item[En entrée :] la liste des sites génériques à contrôler
		\item[En sortie :] La liste des données capteurs envoyée par les sites
	\end{description}
	~\\
	\item réinitialiserSystemeEmbarque(List$<Site>$  s) 
	\begin{description} 
		\item[Rôle :] Permet de réinitialiser les systèmes embarqués
		\item[Contrat :] AUCUN
		\item[En entrée :] La liste des sites génériques pour lesquels l'opération de réinitialisation est demandée
		\item[En sortie :] La listes des sites pour lesquels l'opération n'a pu être validée 
	\end{description}
	~\\
	\item EmettreAlerte(Site s, Alerte a) 
	\begin{description} 
		\item[Rôle :] Permet à un site générique de signaler un dépassement de seuil
		\item[Contrat :] AUCUN
		\item[En entrée :] 
		~\\*
		\begin{itemize}
			\item la référence du site ayant envoyé l'alerte
			\item l'alerte envoyée (cuve vide, capteur défectueux...)
		\end{itemize}
		\item[En sortie :] Aucune sortie
	\end{description}
	~\\

\end{itemize}


Ci-dessous la liste des méthodes utiles selon moi aux APIs de comunication
j'ai mis des attributs pour t'aider mais voilà change si tu veux, c'est toi l'expert des données..
Mais n'hésite pas à m'en parler car j'ai des raisons d'avoir fait ces choix.

Listes des méthodes à rajouter
Communication entre les sites génériques et le système central
    - ConsulterLog(idSite)
    - RecupererDonneesSite (IdSite)
    - SignalerAlerte (IdSite, typeAlerte, details)
    - MAJLogicielle(syteme)
             systeme : RTOS ou Systeme embarqué
    - MaintenanceDistante(typeCommande)
    - ModifierFrequence(idSite, nouvelleFreq)

Communication entre les sociétés de maintenance et les PDAs des camionneurs
    - MAJPlanning (IdSite)
    - ConsulterPlanning(IdSite)
    - SignalerDysfonctionnement (IdSite, typeDysfonctionnement)
    - ConsulterTraitementDysfonctionnement(idSite, typeDysfonctionnement)
            Quand le problème est résolu par notre système, on notifie par mail les responsables
            de la société pour qu'ils puissent demander à leurs employés de faire autre chose 
            si notre solution ne leur convient pas

Communication entre les sociétés de maintenance/ravitaillement et le système central
    - MAJPlanning (IdSite)
    - ConsulterPlanning(IdSite)
    - SignalerDysfonctionnement (IdSite, typeDysfonctionnement)
    - ConsulterTraitementDysfonctionnement(idSite, typeDysfonctionnement)

