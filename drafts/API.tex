\section{API}

\subsection{Communication entre les sociétés de maintenance (ou les PDAs des camioneurs) et le système central}
\begin{itemize}
	\item List$<Planning>$ consulterPlannings()
	\begin{description} 
		\item[Rôle :] Permet à la société de maintenance de récupérer les nouvelles données présentes sur le serveur.
		\item[Contrat :] AUCUN
		\item[En entrée :] AUCUN
		\item[En sortie :] La liste des nouveaux plannings disponibles sur le site central.
	\end{description}
	~\\
	\item mettreAJourPlanning(Planning p)
	\begin{description} 
		\item[Rôle :] Permet à la société de maintenance de mettre à jour le planning (opérations sur site terminée, opérations inutiles...)
		\item[Contrat :] AUCUN
		\item[En entrée :] le planning concerné 
		\item[En sortie :] Aucune sortie
	\end{description}
	~\\
	\item SignalerDysfonctionnement (IdSite, typeDysfonctionnement)
	\begin{description} 
		\item[Rôle :] Permet à la société de maintenance de signaler un dysfonctionnement constaté sur le site indiqué
		\item[Contrat :] AUCUN
		\item[En entrée :] l'identifiant du site et le type de Dysfonctionnement observé
		\item[En sortie :] Aucune sortie
	\end{description}
	~\\
	\item Date ConsulterTraitementDysfonctionnement(idSite, typeDysfonctionnement)
	\begin{description} 
		\item[Rôle :] Permet à la société de maintenance de vérifier si un dysfonctionnement signalé est déjà en cours de traitement
		\item[Contrat :] AUCUN
		\item[En entrée :] l'identifiant du site et le dysfonctionnement à vérifier
		\item[En sortie :] La date prévisionnelle de réparation, ou une valeur telle que 00/00/0000 si pas de traitement en cours
	\end{description}
	~\\
	\item AdresseGPS coordonneesGPS(Site s)
	\begin{description} 
		\item[Rôle :] Permet de récupérer les coordonnées GPS d'un site passé en paramètre
		\item[Contrat :] Le site doit exister
		\item[En entrée :] un site générique
		\item[En sortie :] les coordonnées GPS de ce site
	\end{description}
	~\\
\end{itemize}

\subsection{Communication entre les sites génériques et le système central}
\begin{itemize}
	\item StringList ConsulterLog(idSite)
	\begin{description} 
		\item[Rôle :] Permet de consulter toutes les opérations enregistrées dans le fichier log du système embarqué
		\item[Contrat :] AUCUN 
		\item[En entrée :] l'identifiant du site dont on souhaite connaitre le log
		\item[En sortie :] Le log sous forme de liste de chaines de caractères
	\end{description}
	~\\
	\item MAJLogicielle(syteme)
	\begin{description} 
		\item[Rôle :] Permet de mettre à jour l'OS du système embarqué.
		\item[Contrat :] AUCUN 
		\item[En entrée :] Le nouveau système (le fichier directement envoyé)
		\item[En sortie :] Aucune sortie
	\end{description}
	~\\
	\item MaintenanceDistante(typeCommande, detail)
	\begin{description} 
		\item[Rôle :] Permet d'efffectuer des opérations de maintenance à distance sur le système embarqué
		\item[Contrat :] AUCUN 
		\item[En entrée :]
		 ~\\*
		\begin{itemize}
			\item la commande à effectuer (réinitialiser un capteur, effacdr le log...)
			\item les paramètres particuliers de la commande effectuée (le capteur à réinitialiser...)
		\end{itemize}
		\item[En sortie :] Aucune sortie
	\end{description}
	~\\
	\item ModifierFrequence(idSite, nouvelleFreq)
	\begin{description} 
		\item[Rôle :] Permet de définir la fréquence à laquelle le système embarqué enverra les nouvelles données des capteurs
		\item[Contrat :] AUCUN 
		\item[En entrée :] l'identifiant du site pour lequel on définit cette fréquence, la nouvelle fréquence à appliquer
		\item[En sortie :] Aucune sortie
	\end{description}
	~\\
	\item List$<RapportSite>$ recupererDonneesSite(Site) 
	\begin{description} 
		\item[Rôle :] Permet de rafraichir l'état des capteurs des sites passés en paramètre et de récupérer les données actuelles instantanément.
		\item[Contrat :] AUCUN 
		\item[En entrée :] Le site générique à contrôler
		\item[En sortie :] La liste des données capteurs envoyée par le site
	\end{description}
	~\\
	\item réinitialiserSystemeEmbarque(List$<Site>$) 
	\begin{description} 
		\item[Rôle :] Permet de réinitialiser les systèmes embarqués
		\item[Contrat :] AUCUN
		\item[En entrée :] La liste des sites génériques pour lesquels l'opération de réinitialisation est demandée
		\item[En sortie :] La listes des sites pour lesquels l'opération n'a pu être validée 
	\end{description}
	~\\
	\item SignalerAlerte (IdSite, typeAlerte, details)
	\begin{description} 
		\item[Rôle :] Permet à un site générique de signaler un dépassement de seuil
		\item[Contrat :] AUCUN
		\item[En entrée :]
		~\\*
		\begin{itemize}
			\item la référence du site ayant envoyé l'alerte
			\item l'alerte envoyée (cuve vide, capteur défectueux...)
			\item un troisième paramètre spécifiant des détails relatifs au type d'alerte (quel capteur est défectueux...)
		\end{itemize}
		\item[En sortie :] Aucune sortie
	\end{description}
	~\\

\end{itemize}


%Ci-dessous la liste des méthodes utiles selon moi aux APIs de comunication
%j'ai mis des attributs pour t'aider mais voilà change si tu veux, c'est toi l'expert des données..
%Mais n'hésite pas à m'en parler car j'ai des raisons d'avoir fait ces choix.

%Listes des méthodes à rajouter
%Communication entre les sites génériques et le système central
%    - ConsulterLog(idSite)
%    - RecupererDonneesSite (IdSite)
%    - SignalerAlerte (IdSite, typeAlerte, details)
%    - MAJLogicielle(syteme)
%             systeme : RTOS ou Systeme embarqué
%    - MaintenanceDistante(typeCommande)
%    - ModifierFrequence(idSite, nouvelleFreq)
%
%Communication entre le site central et les PDAs des camionneurs
%    - MAJPlanning (IdSite)
%    - ConsulterPlanning(IdSite)
%    - SignalerDysfonctionnement (IdSite, typeDysfonctionnement)
%    - ConsulterTraitementDysfonctionnement(idSite, typeDysfonctionnement)
%            Quand le problème est résolu par notre système, on notifie par mail les responsables
%            de la société pour qu'ils puissent demander à leurs employés de faire autre chose 
%           si notre solution ne leur convient pas
%
%Communication entre les sociétés de maintenance/ravitaillement et le système central
%    - MAJPlanning (IdSite)
%    - ConsulterPlanning(IdSite)
%    - SignalerDysfonctionnement (IdSite, typeDysfonctionnement)
%    - ConsulterTraitementDysfonctionnement(idSite, typeDysfonctionnement)

