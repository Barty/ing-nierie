Les principales exigences non fonctionnelles sont (classement par priorité décroissante) : 
- Fiabilité
- Autonomie
- Robustesse (limitations technlogique)
- Générécité (evolutivite, maintenabilite, reutilisation)
- Intégration de l'existant
- Ergonomie
- Traçabilité


==>		Faut-il le tableau bizarre du choix de la priorité?		<==


- Fiabilité

- Autonomie
	Ce critère apparaît clairement au vu des besoins du nouveau système comme une réelle exigence non fonctionnelle.
En effet les systèmes distant doivent pouvoir fonctionner de façon autonome et sur de longues périodes (voir 1an)
pour réduire au maximum les inverventions sur les sites isolés. Cette exigence fonctionnelle est critique pour le bon
fonctionnement de notre système.

- Robustesse

- Générécité
	Bien que la mise en place de notre système découle d'un besoin spécifique (surveillance à distance de sites 
isolés), notre système doit être adaptable à d'autres besoins avec un minimum de modifications à y apporter. De ce 
fait, durant toutes les phases de spécifications/conception ainsi que de réalisation, la solution qui devra être 
proposée ainsi que ces différentes briques doit être conçus dans un but de réutilisation. En effet notre système 
devra être adaptable pour aboutir à des applications du type : surveillance de trains, surveillance de vieux lol..

- Intégration de l'existant:

Le système devrait s'adapter aux processus existants qui seront partiellement/totallement conservés.
La re-utilisation des processus sera aussi un atout pour le nouveau système...???

- Ergonomie:  Les utilisateurs sont les non-informaticiens, alorrs les IHM 
développées devront être intuitive, facile à utiliser. Egalement, ils doivent être adaptés aux
différents périphériques (PDA, PC).   

- Traçabilité: 
il est indispensable de sauvegarder une trace de toute activité effectuée par les utilisateurs connectés
au système. Il est aussi important de garder une historique de toute opération maintenance, les logs d'erreur
pour trouver la source de dysfonctionnement....