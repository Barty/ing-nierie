\documentclass{article}

\begin{document}


\subsection{SYSTEME CENTRAL}
		Son role est de collecter les informations des differents sites et de permettre l'exploitation de ces dernieres.
		Des contraintes apparaissent clairement : 
		\begin{description}
				\item[Fiabilite/Surete de fonctionnement :] afin d'exploiter correctement les donnees et d'envoyer en temps reel des commandes aux sites distants si necessaire.
				\item[Tracabilite :] sur 2ans au moins des informations provenant des sites afin de gerer correctement les reprises sur erreur, apporter les corrections necessaires en cas de fausses manipulations humaines ou de dysfonctionnement temporaire du systeme.
				\item[Efficacite :] proposer un ensemble de services permettant le pilotage et le monitoring a distance des systemes distants.
		\end{description}

		Une des alternatives pour le systeme central serait d'utiliser le systeme \textbf{Data Management System} de AXYS Technologies INC qui integre avec les systemes Watchman500 Node et des capteurs, permet le pilotage a distance des Watchman500 Nodde, la collecte des donnees, ainsi que leur exploitation.
		Cependant, ce qui apparait en premier lieu, serait de mettre en place une solution specifique qui convienne le mieux a nos besoins, mais il ne faut pas negliger le fait que cette alternative est tres couteuse.
		Concernant la collecte des donnees, on peu observer qu'elles semblent etre de petites tailles donc on pourrait s'otienter vers un SGBD tel que MySQL. Ou plutot dans une optique d'evolution et de genericite, on pourrait essayer de prevoir des montees en charge des donnees et faire le choix de systemes tels que Orcal, SQL Server.
		Pour la collecte des donnees une autre solution serait de la confier a des hebergeurs professionnels, des solutions tres interessantes existent sur le marche, par exemple la solution \textbf{Windows Select Server} de Orange Business Service. Cette solution a de gros avantages supplementaires :
		\begin{itemize}
				\item Pas de maintenance des serveurs, un srvice de maintenance professionnel sous 4h pour moins de 140euros HT/mois.
				\item Donc pas besoin d'�quipe de maintenance technique.
		\end{itemize}


\end{document}
