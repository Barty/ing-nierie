\documentclass{article}

\begin{document}




\subsection{ANALYSE DE L'EXISTANT}

		Actuellement le savoir-faire est tr�s limit�, en effet il y a peu de processus de gestion d�clench�s pour surveiller les sites. Le sch�ma ci-dessous r�sume la surveillance actuelle.
\newline

					INSERER LE SCHEMA

\newline

		Le mat�riel est celui utilis� par les soci�t�s de maintenance intervenant sur les sites (camions, syst�mes de vidage/remplissage..)
Deux m�tiers ont �t� identifi�s, les propri�taires des sites s�occupent de leur exploitation tandis que les soci�t�s sp�cialis�s sont charg�es du m�tier de la maintenance.

AUBRY ==> Petit bilan sur les dysfonctionnement remarques et deja les ameliorations � definir!

	Plusieurs dysfonctionnements ont ete remarques des ici : 
	- la non regularite des surveillances par les exploitants (proprietaires des sites)
	- La non assurance de la maintenance directement par une societe quand elle est contactee car son parc ne dispose pas de camions disponibles.
	- les risques de d�sastres humains et environnementaux encourus.
	
	Le nouveau systeme devra remedier aux problemes detectees en : 
	- autonome pour ne pas necessiter d'interventions humaines regulieres pour des operations de constat ou de maintenance
	- generique car sera implente sur differents sites
	- pilotable a distance (configuration, commande..)
	- fiable vu les enjeux environnements en jeu
	
	Par rapport au systeme actuel il permettra une surveillance temps reel a partir d'un poste distant. 
	Il permettra egalement de r�duire considerablement les interventions et d'avoir une meilleur planification de celles-ci, afin de r�aliser des economies logistiques tout en garatissant certaines autres exigences (criticite, meilleur surveillance..).
	
\end{document}