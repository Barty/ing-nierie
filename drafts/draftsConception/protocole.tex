\section{Règles}
    \subsection{spécification du protocole de communication}

        Le protocole doit pouvoir être intérrogé pour fournir des valeurs correspondant à un ou plusieurs parametres.
        Il doit pouvoir fournir un historique des valeurs, ou dire qu'aucun historique n'est disponible.
        (Tout ce que le protocole peut faire, n'est pas forcement ce que les capteurs devront savoir faire).

        Le protocole doit permettre de demander au capteur d'envoyer réguliérement , à chaque modification, ou de n'envoyer que sur demande la valeurs correpondant à un parametre.
        En cas de demande sur modification, un seuil de modification doit être précisé.
        Le capteur peut ne pas être capable d'envoyer la valeur sur modification, ou même à la demande.
        Il peut avoir été conçus pour envoyer réguliérement une valeur, afin de simplifier au maximum la technologie du capteur.

        Le capteur peut demander un accusé de reception de la valeur.

        Chaque capteur possède une adresse pour l'accés, et éventuellement un nom.
        Le nom du capteur s'acquiert de la même façon que n'importe quel parametre.
        (Il serait stupide mais pas impossible de demander d'envoyer réguliérement le nom du capteur)

        Le protocole permet la modification de valeurs (par exemple le nom, ou plus tard, peut être des valeurs de configuration etc ...)
        L'historique devrait pouvoir retenir qui à modifier la valeur (quel capteur, ou message externe).

        Le protocole devrait être sécurisé.

        Toutes communication avec un terminal sera borné à l'adresse du client et l'adresse du capteur.
        Pour initialiser une nouvelle communication, un code securisant sera demandé.
        Un capteur peut gérer, une ou plusieurs connections.

        Un site peut être le relais d'un autre site plus petit, plus faible technologiquement.
        Dans ce cas, les parametres du site plus petit sont cloné sur le site relais.
        La communication entre le site "esclave" et le site relais peut se faire par un autre media que le media entre le site relais et le client.

        Un capteur peut être initialement configuré pour ne communiquer qu'avec une adresse spécifié "en dur".

        La localisation, comme le nom peut être intérrogé comme tout autres paramètre.

        Un paramètre peut avoir une valeur sous forme de chaine de caractere, de valeur numérique, ou de suite de valeur numérique.


        Toutes actions sur la station peut se faire à l'aide de paramètre spécifique (par exemple : vitesse moteur = 10 )
        ou bien par l'intermédiaire d'un parametre de commande (par exemple : commande = exec vitesse_moteur 10)

        Le protocole doit pouvoir spécifier la fréquence d'acquisition des capteurs.

        %TODO TROUVER UN MODEL GENERIQUE capteur -> site -> client


\subsection{}

