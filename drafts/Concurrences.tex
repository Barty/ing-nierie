\section{Concurrence}

Sur le marché européen notre société tient encore largement sa place. Cependant, des sociétés concurrentes existent et ont toujours une part importante du marché. Les principales sont : National Instruments et AllianTech.

        \subsection{Axys Technologies INC}
        AXYS Technologies Inc (AXYS) est une compagnie canadienne ayant plus de 30 ans d'expérience dans la conception, la fabrication et l'installation de systèmes de télésurveillance de l'environnement à travers le monde. AXYS applique ses connaissances approfondies dans divers domaines : celui marin, dans des stations de surveillance terrestres et offshore, dans des systèmes d'évaluation des ressources éoliennes de mesure des paramètres aquatiques, océaniques et atmosphériques. La société propose également les services techniques sur le terrain pour former et soutenir les clients dans l'exploitation et la maintenance de tous les produits. Le système embarqué que nous préconisons pour notre système est d'ailleurs fabriqué par leur compagnie. Leur système semble être le plus abouti, aujourd'hui le système embarqué qu'il fabrique est l'un des meilleurs du marché. Il propose une solution comprenant les 3 composants suivants :
\begin{description}
           \item[Watchman500 Node :] Permettant de collecter les données des capteurs
           \item[Data Management System (DMS) :] Permettant fr configurer le système embarqué.
           \item[Smart Web et Smart View :] Sont une application web et un logiciel permettant de superviser à distance le site et d'exploiter les données recueillis. Ils permettent également d'exporter les données sous divers formats (CSV, XMl, Excel..)
\end{description}

        \subsection{National Instruments}
        Il semble être l'un des leaders sur notre domaine d'activité. La société est basé à Austin au Texas, elle a été créée en 1976. Elle a ses filiales dans plus de quarante pays. Un aspect qui nous permet de les concurrencer est qu'aucune industrie ne représente plus de 10% de leurs ventes, en effet ils ont des activités très variées. Il investisse 16% de leur chiffre d'affaires annuel dans la recherche et le développement et sorte en moyenne 208 produits par an. Ce qui fait d'eux, assez souvent, un précurseur dans leurs domaines d'activités. 
        Il propose un système personnalisable appelé WSN très semblable à celui que nous mettrons en place, qui permet également la surveillance en extérieur d'équipements.

        \subsection{AllianTech}
        La société a été créée le 27 Septembre 1999, par d’anciens cadres de la Société Endevco France. Du fait de l’expérience de ses principaux fondateurs et salariés, s’est assurée la confiance de nombreux fournisseurs et clients. La société dispose de solutions complètes dans les domaines de la mesure de vibration, de l’analyse et de la surveillance vibratoire, des mesures de chocs, du pesage, de l’acquisition et de l’enregistrement de données ainsi que des systèmes d’étalonnage d’accéléromètres portables. 

