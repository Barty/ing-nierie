\documentclass{article}

\begin{document}




\subsection{ANALYSE DE L'EXISTANT}

		Actuellement le savoir-faire est très limité, en effet il y a peu de processus de gestion déclenchés pour surveiller les sites. Le schéma ci-dessous résume la surveillance actuelle.
\newline

					INSÉRER LE SCHÉMA

\newline

		Le matériel est celui utilisé par les sociétés de maintenance intervenant sur les sites (camions, systèmes de vidage/remplissage..)
Deux métiers ont été identifiés, les propriétaires des sites s'occupent de leur exploitation tandis que les sociétés spécialisés sont chargées du métier de la maintenance.

AUBRY ==> Petit bilan sur les dysfonctionnement remarques et déjà les améliorations é définir!

	Plusieurs dysfonctionnements ont été remarques des ici : 
	- la non régularité des surveillances par les exploitants (propriétaires des sites)
	- La non assurance de la maintenance directement par une société quand elle est contactée car son parc ne dispose pas de camions disponibles.
	- les risques de désastres humains et environnementaux encourus.
	
	Le nouveau système devra remédier aux problèmes détectées en : 
	- autonome pour ne pas nécessiter d'interventions humaines régulières pour des opérations de constat ou de maintenance
	- générique car sera implémenté sur différents sites
	- pilotage a distance (configuration, commande..)
	- fiable vu les enjeux environnements en jeu
	
	Par rapport au système actuel il permettra une surveillance temps réel a partir d'un poste distant. 
	Il permettra également de réduire considérablement les interventions et d'avoir une meilleur planification de celles-ci, afin de réaliser des économies logistiques tout en garantissant certaines autres exigences (criticité, meilleur surveillance..).
	
\end{document}
