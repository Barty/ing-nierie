% court résumé

\section{Qualité de service : optimisation de l'itinéraire à proposer à la société de maintenance}

    L'architecture complète permettras de connaître précisement en temps et en quantité l'état des cuves à vider ou remplir.
    C'est le moyen de calculer de manière trés efficace un itinaire le plus court possible et le plus efficace, afin de minimiser les consommation et les temps de conduite.
    
    \subsection{Optimisation du trajet}
        
        Le but est de minimiser le trajet, il faudras donc regrouper les lieux à traiter par distance géographique et temporelle.
        La suite logicielle du site centrale récupereras la totalité des informations concernant les sites isolés, c'est donc dans cette suite logicielle qu'il faudra intégrer un module d'optimisation de traitement des sites isolés.
        \\
        Au lieu de soumettre directement les lieux et les dates des sites à traiter à la société de maintenance, il sera possible de soumettre directement un itinéraire précalculé et optimisé.
        
        
% Cahier des charges

\section{Introduction}

        Dans le cadre de la rénovation du système d'information visant à réduire l'impact écologique, il peut être enviseagable et trés avantageux de mettre en place un système d'optimisation des trajets de ravitaillement et de mainteance.

    \subsection{Présentation du projet}
    
        Ce projet d'optimisation s'inscrit immédiatement dans un projet plus vaste d'amélioration des flux d'information et de rénovation du systéme d'information.
        
        Ce projet est initié ... %TODO
        
        Cette solution se placera à la bordure du système avec la communication envers la société en charge des déplacement de ravitaillement et de maintenance.
        
    \subsection{Présentation du document}
        
        %TODO
        
    \subsection{Document applicables / Documents de référence}
    
    \subsection{Terminologie et abréviations}
    
\section{Présentation du problème}
        
        Un des but recherché dans l'améliorétion du système d'information est la réduction de l'impact écologique engendré. Et c'est dans ce but précis qu'à été conçus et réfléchie cette solution.
        
    \subsection{But}
    
        Le but de cette solution est de réduire l'impact écologique en agissant sur la gestion des trajet de ravitaillement et de maintenance. En agençant au mieux les étapes des trajets il sera possible de réduire la distance parcouru.
        
    \subsection{Formulation des besoins, exploitation et ergonomie, expérience}
    
        \subsubsection{Formulation des besoins}
            
            Le besoin exprimé est directement lié aux besoin du renouvellement de l'installation. Il s'agit de réduire l'impact écologique des activités lié au ravitaillement et à la maintenance.
            
        \subsubsection{Exploitation}
        
            
        
        \subsubsection{Ergonomie}
        
        \subsubsection{Expérience}
    
    \subsection{Portés, développement, mise en oeuvre, organisation de la maintenance}
    
    \subsection{Limites}
    
        Les limites de ce système ce situe dans la communication de l'itinéraire optimisé à la société chargé du ravitaillement et de la maintenance.
        Si l'itinéraire est calculé avant la communication, le problème qui se pose est l'impossibilité pour l'entreprise de le modifier.
        Pour illustrer ce problème, prenons l'exemple d'une impracabilité routiére altérant l'itinéraire calculé. Il est alors possible q'un recalcul d'itinéraire soit plus efficace et conduise à un itinéraire alternatif plus optimisé que l'itinéraire altérnatif proposé par les services routier.
        Une autre illustration peut s'envisager si la société chargé du ravitaillement et de la maintenance s'occupe en paralléle de sites différents.
        
\section{Exigences fonctionnelles}

    \subsection{Fonction de base, performances et aptitudes}

    \subsection{Contraintes d'utilisation}
    
    \subsection{Critére d'appréciation de la réalisation effective de la fonction}
    
        Le premier critére d'appréciation immédiatement enviseagable est la différence de distance entre les itinéraire avant et après optimisation.
        Un second critére plus effectif serait la différence de consommation réelle de carburant qui est plus liée à l'impact écologique.
    
    \subsection{Flexibilité dans la façon de mettre en \oeuvre la fonction concernée et variation de coûts associée en fonction de cette flexibilité}

\section{Contraintes imposées, faisabilité technologique et éventuellement moyens}

    \subsection{Sûreté, planning, organisation, communication}
    
    \subsection{Complexité}
    
    \subsection{Compétences, moyens et règles}
    
    \subsection{Normes de documentation}

\section{Configuration cible}

    \subsection{Matériel et logiciels}
    
    \subsection{Stabilité de la configuration}
    
    \subsection{Normes de documentation}

\section{Guide de réponse au cahier des charges}

    \subsection{Grille d'évaluation}

\section{Annexes}

    ...
