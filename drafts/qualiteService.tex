% court résumé

\section{Qualité de service : optimisation de l'itinéraire à proposer à la société de maintenance}

    L'architecture complète permettras de connaître précisement en temps et en quantité l'état des cuves à vider ou remplir.
    C'est le moyen de calculer de manière trés efficace un itinaire le plus court possible et le plus efficace, afin de minimiser les consommation et les temps de conduite.
    
    \subsection{Optimisation du trajet}
        
        Le but est de minimiser le trajet, il faudras donc regrouper les lieux à traiter par distance géographique et temporelle.
        La suite logicielle du site centrale récupereras la totalité des informations concernant les sites isolés, c'est donc dans cette suite logicielle qu'il faudra intégrer un module d'optimisation de traitement des sites isolés.
        \\
        Au lieu de soumettre directement les lieux et les dates des sites à traiter à la société de maintenance, il sera possible de soumettre directement un itinéraire précalculé et optimisé.
        
        
% Cahier des charges

\section{Introduction}

    \subsection{Présentation du projet}
    
    \subsection{Présentation du document}
    
    \subsection{Document applicables / Documents de référence}
    
    \subsection{Terminologie et abréviations}
    
\section{Présentation du problème}

    \subsection{But}
    
    \subsection{Formulation des besoins, exploitation et ergonomie, expérience}
    
    \subsection{Portés, développement, mise en oeuvre, organisation de la maintenance}
    
    \subsection{Limites}

\section{Exigences fonctionnelles}

    \subsection{Fonction de base, performances et aptitudes}

    \subsection{Contraintes d'utilisation}
    
    \subsection{Critére d'appréciation de la réalisation effective de la fonction}
    
    \subsection{Flexibilité dans la façon de mettre en \oeuvre la fonction concernée et variation de coûts associée en fonction de cette flexibilité}

\section{Contraintes imposées, faisabilité technologique et éventuellement moyens}

    \subsection{Sûreté, planning, organisation, communication}
    
    \subsection{Complexité}
    
    \subsection{Compétences, moyens et règles}
    
    \subsection{Normes de documentation}

\section{Configuration cible}

    \subsection{Matériel et logiciels}
    
    \subsection{Stabilité de la configuration}
    
    \subsection{Normes de documentation}

\section{Guide de réponse au cahier des charges}

    \subsection{Grille d'évaluation}

\section{Annexes}

    ...
